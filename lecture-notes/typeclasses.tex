\section{Ad-Hoc Polymorphism with Typeclasses}

\begin{notelist}
    \item At the machine level, adding two integers is quite a different operation from
          adding two floating-point numbers. High-level languages, in an effort to hide
          this detail, \textit{overload} the semantics of the addition operator to support 
          these distinct operations using the same operator. 

    \item In a strongly-typed language like Haskell, what might the type of \code{(+)} be?
    \begin{notelist}
        \item In Haskell, infix binary operators are just syntactic sugar for functions
              of two arguments. When referring to binary operators outside of their
              normal infix notation, Haskell requires them to be surrounded by parentheses.
    \end{notelist}

    \item \code{Int -> Int -> Int} or similar is insufficient, since the type of \code{(+)}
          needs to be general enough to describe adding together two operands
          of many different numeric types.

    \item \code{a -> a -> a} seems promising: two operands and a result, all of the same type. However,
          this type signature \textit{unifies} with \code{TravelDetails -> TravelDetails -> TravelDetails}
          and addition of that type does not make sense.

    \item What happens if we ask the Haskell compiler to infer the type of \code{(+)}?
    \begin{notelist}
        \item The most popular Haskell compiler, GHC, has a REPL interface called GHCi.
        \item The command \code{:t \textit{expression}} asks GHCi to infer the type of
              \code{\textit{expression}}.
        \item \code{:t (+)} yields the inferred type: \code{(Num a) => a -> a -> a}.
    \end{notelist}

    \item \code{(Num a) => ...} is a \textit{class constraint} on the type signature that
          follows the \code{=>}. 

    \item Normally, a free type variable in a type signature can be unified with any type at all.
    
    \item A class constraint on a type variable restricts the types that it can be unified with
          to types that are \textit{instances} of the named \textit{type class}.

    \item So \code{(+) :: (Num a) => a -> a -> a} says that \code{(+)} is a function of
          two operands and result all of some type \code{a} \textit{where \code{a} is
          an instance of the type class \code{Num}}.

    \item A type class acts a bit like a \textit{interface} in object-oriented programming.
          It acts as a contract: any type that is an instance of a type class must implement 
          certain methods to qualify.

    \item The terminology may be a bit confusing:
    \begin{notelist}
        \item In an OO language, an object is an \textit{instance} of a \textit{class} which
              might \textit{implement} an \textit{interface}.
        \item In Haskell, a value \textit{has} a \textit{type} which might be an \textit{instance}
              of a \textit{type class}.
    \end{notelist}
\end{notelist}

\section{Basic Typeclasses from The Haskell Prelude}

Haskell offers quite a bit of functionality in its standard library. In particular, the \textit{Prelude},
the set of type and function definitions imported automatically into every program, defines 

One of the most basic typeclasses is \code{Eq}, consisting of types that implement an equality-testing
operation. Here is how it is defined:

\begin{lstlisting}
  class Eq a where
        (==), (/=)  ::  a -> a -> Bool

        x /= y  = not (x == y)
        x == y  = not (x /= y)
\end{lstlisting}

\begin{notelist}
    \item The \code{class} keyword introduces a type class definition, followed by the
          name of the type class and a type variable that we will use in the description
          of the type class's interface. Think of this type variable as a formal parameter
          in a function definition. 

    \item The \code{Eq} type class defines two required operations, equality and inequality.
          In this case, the two have exactly the same signature, so the type annotation
          is shared.

    \item If we asked Haskell to infer the type of \code{(==)}, what would we get?
          \code{(==) :: (Eq a) => a -> a -> Bool}.

    \item Then we see two equational function definitions. These are default implementations
          for \code{Eq}'s operations.

    \item This means we do not have to define both \code{(==)} and \code{(/=)}. Each is defined
          in terms of the other, so an implementation for one is enough. The compiler will
          complain if neither is implemented.

    \item Because we can define default implementations, type classes are actually more like
          \textit{abstract classes} in OO languages.
\end{notelist}

Typeclasses can themselves have class constraints. Here is the definition of \code{Ord}, which describes
operations available for totally ordered data types:

\begin{lstlisting}
class Eq a => Ord a where
    compare              :: a -> a -> Ordering
    (<), (>=), (>), (<=) :: a -> a -> Bool
    max                  :: a -> a -> a
    min                  :: a -> a -> a

data Ordering = LT | EQ | GT
\end{lstlisting}

\begin{notelist}
    \item Class constraints in a type annotation, as in line 1 above, require that the constrained
          type variable be an instance of the given typeclass.
    
    \item In this case, for a type to be an instance of \code{Ord}, it must also be an instance of
          \code{Eq}. It should be pretty clear why that is necessary.
    
    \item The full definition of \code{Ord} gives default implementations for all these operations so
          that an instance need only implement either \code{compare} or \code{(<=)}.
\end{notelist}

There are several other typeclasses worth mentioning:

\begin{notelist}
    \item \code{Show} instances can be turned into a \code{String} representation with \code{show :: (Show a) => a -> String}.
    \item \code{Read} instances know how to undo the process and turn a string into a value.
    \item \code{Bounded} instances are types with a smallest and largest values, given as two polymorphic 
          constants \code{minBound, maxBound :: (Bounded a) => a}.
    \item \code{Enum} instances are sequentially ordered types. Given a value in that sequence, we can use
          \code{succ, pred :: (Enum a) => a -> a} to get the next or previous value. We can use 
          \code{enumFromTo :: (Enum a) => a -> a -> [a]} to get a list containing the elements in the sequence
          between a start value and an end value, inclusive.
\end{notelist}

Haskell's standard library also defines a hierarchy of numeric typeclasses.

\begin{notelist}
    \item We saw that GHC would infer the type \code{(Num a) => a -> a -> a} for the \code{(+)} operator.
          That, along with \code{(*)}, \code{(-)} (the binary subtraction operator), \code{negate} (for unary
          negation), and a couple of others define the most basic interface for numeric types.
    \item \code{Fractional} extends \code{Num} with division in \code{(/)} and reciprocation in \code{recip}.
    \item \code{Floating} extends \code{Fractional} with real-valued logarithms, exponentiation, trigonometric
          functions and even \code{(Floating a) => pi :: a}, the polymorphic constant $\pi$.
\end{notelist}

The full numeric hierarchy is even richer and there is plenty of detail in the Prelude's typeclasses that we
have glossed over. Full details are available in \cite[section 6.4]{haskell98}.

\section{Creating New Typeclass Instances}

Haskell typeclasses are \textit{open}, meaning that we can define new instances of typeclasses defined in the
Prelude or elsewhere.

Let's see how we can implement some of the Prelude's basic typeclasses for a simple type.

\begin{lstlisting}
data Section = Coach
             | Business
             | FirstClass

instance Eq Section where
    FirstClass == FirstClass = True
    Business   == Business   = True
    Coach      == Coach      = True
    _          == _          = False

instance Ord Section where
    x          <= y          | x == y = True
    Coach      <= _                   = True
    Business   <= Coach               = False
    Business   <= FirstClass          = True
    FirstClass <= _                   = False
    
instance Show Section where
    show Coach      = "Coach"
    show Business   = "Business"
    show FirstClass = "FirstClass"
\end{lstlisting}

\begin{notelist}
    \item An instance declaration begins with the keyword \code{instance}, followed by equational
          definitions for the various functions defined for the class.
    \item The definition of \code{(==)} for \code{Eq} is straightforward. We define the function
          in four cases. In the first three equations, we enumerate the cases where values could
          be considered equal and the last equation is a catch-all: the underscore character 
          matches any value, so any case not matched by the first three equations will get caught
          by the fourth and will return \code{False}.
    \item We define an \code{Ord} instance by enumerating the ways in which \code{Section} values
          can be ordered. The first equation uses a \textit{guard}: \code{x} and \code{y} will match
          any values, but the match is only successful if the \textit{guard expression} evaluates
          to \code{True}.
    \item The \code{Show} instance is trivial: we simpliy define a string value to return for each
          of \code{Section}'s three data constructors.
\end{notelist}

We can imagine that defining instances for these typeclasses would be quite similar for any algebraic
data type. It seems trivial to automatically construct an \code{Eq} instance for any simple sum type.
Furthermore, because of the recursive nature of algebraic data types, it would be easy to extend that
idea to arbitrary sum-of-products types. 

\begin{lstlisting}
data S = P1 T1_1 ... T1_K1
       | P2 T2_1 ... T2_K2
         ...
       | PN TN_1 ... TN_KN

instance Eq S where
    P1 u1_1 ... u1_k1 == P1 v1_1 ... v1_k1 = u1_1 == v1_1 && .. && u1_k1 == v1_k1
    P2 u2_1 ... u2_k2 == P2 v2_1 ... v2_k2 = u2_1 == v1_1 && .. && u2_k2 == v2_k2
    ..
    PN un_1 ... uN_kN == PN vN_1 ... v1_kN = uN_1 == vN_N && .. && uN_kN == vN_kN
    _ == _ = False
\end{lstlisting}

\begin{notelist}
    \item This is pseudocode for the general form of an \code{Eq} instance for a sum of
          $N$ constructors that are each a product of $K_N$ values.
    \item In words, two \code{S} values are equal if their data constructors are equal and
          each pair of constituent values are equal.
\end{notelist}

In fact, Haskell offers the ability to \textit{derive} typeclass instances, and not just for 
\code{Eq}.

\begin{lstlisting}
data Section = Coach
             | Business
             | FirstClass
             derving (Eq, Ord, Show)
\end{lstlisting}

\begin{notelist}
    \item The \code{deriving} keyword instructs the compiler to automatically derive instances
          for the typeclasses that follow.
    \item The Haskell 98 standard can derive instances for \code{Eq}, \code{Ord}, \code{Enum},
          \code{Bounded}, \code{Show}, and \code{Read}.
    \item In automatically derived instances of \code{Ord}, \code{Enum}, and \code{Bounded},
          the order of declaration of the data constructors is used. So our derived
          \code{Ord} instance for \code{Section} still returns \code{True} for  \code{Coach <= FirstClass}.
\end{notelist}

Because the derived definitions are recursive, we might not always be able to derive instances
when the constituents of product types do not support the operations we need:

\begin{lstlisting}
data Section = Coach
             | Business
             | FirstClass BeverageOption
             derving (Eq, Ord, Show)

data BeverageOption = Wine
                    | Beer
                    | Soda
\end{lstlisting}

\begin{notelist}
    \item Here, we cannot automatically derive an \code{Eq} instance for \code{Section} because 
          \code{BeverageOption} is not an instance of \code{Eq}. We cannot determine if two
          values using the \code{FirstClass} constructor are equal because we have no way of
          checking two \code{BeverageOption} values for equality.
    \item Similarly, we could not derive an \code{Ord} instance since we have no notion of
          ordering on \code{BeverageOption}s.
    \item In this case, adding a \code{deriving} clause to our definition of the
          \code{BeverageOption} type would resolve the problem.
\end{notelist}
