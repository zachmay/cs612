\section{Basic Haskell Types}

In this section, we take a deeper look at Haskell's type system.

\begin{notelist}
\item Haskell offers the some basic primitive data types we would expect: \cite{haskell98}
\begin{notelist}
    \item \code{Int} and \code{Integer}: machine-sized and arbitrary precision integers, respectively.
    \item \code{Float} and \code{Double}: single- and double-precision floating point numbers.
    \item \code{Char}: Single Unicode characters.
    \item \code{Bool}: Boolean values (but see below, \code{Bool} is actually a composite type).
\end{notelist}

\item Additionally, we have several composite data types:
\begin{notelist}
    \item \code{cons} lists: Lisp-style linked lists. The type is written \code{[a]} where \code{a}
          is another type.
    \item \code{String}: character strings; \code{String} is literally just a \textit{type synonym} for \code{[Char]}.
    \item Tuples: $k$-element tuples, $k \geq 2$. The type of an $n$-element tuple is \code{(a1, a2, ..., an)} where
          \code{a1}, \code{a2}, \ldots, and \code{an} are other types.
\end{notelist}

\item New data types are introduced with the \code{data} keyword.

\item Type synonyms can be introduced with the \code{type} keyword. \code{type EmailAddress = String} says that 
      the identifier \code{EmailAddress} can be used interchangeably with the type \code{String}. The advantage
      is that \code{EmailAddress} is more descriptive.

\item The keyword \code{newtype} creates a more controlled sort of type synonym.
\begin{notelist}
    \item If we wanted a type to describe e-mail address values but did not want it to be
          interchangeable with \code{String}s in general, we could define a new type that simply ``tags''
          a \code{String} value: \code{data EmailAddress = EmailAddress String}.
    \item This comes with some amount of overhead each time we want to ``unwrap'' the \code{EmailAddress} and
          get at the underlying \code{String}.
    \item Instead we can use \code{newtype EmailAddress = EmailAddress String}. Haskell's type checker treats
          this exactly like type introduced with \code{data}, but drops the tagging for purposes of code
          generation, eliminating the overhead required to ``unwrap'' the \code{String}.
\end{notelist}

\item Although Haskell can \textit{infer} the types of most all expressions, types can be stated explicitly
      with a type annotation using \code{::}. For example:
\begin{lstlisting}
nothing :: [String]
nothing = []

moreNothing = []
\end{lstlisting}
\begin{notelist}
    \item We define two values, \code{nothing} and \code{moreNothing}.
    \item Although the equational definitions are identical, we explicitly define the type of
          \code{nothing} to be of type \code{[String]}, a list of strings, with the type
          annotation on line 1.
    \item Without an explicit type annotation, Haskell will infer the type of \code{moreNothing},
          in this case, the more general type \code{[a]}. (This is a \textit{polymorphic type} which
          we will discuss later.)
\end{notelist}

\item Haskell's lists and tuples are specific examples of the language's \textit{algebraic type system}.
\item Algebraic data types were introduced in the Hope programming language in 1980. \cite{hope}
\item An algebraic type system generally offers two sorts of types:
\begin{notelist}
    \item \textit{Product types}: A data type with one or more fields.
    \begin{notelist}
        \item Tuples are the archetypal product type.
        \item The ``size'' of a product type is the product of the sizes of the types of its fields.
        \item E.g., \code{(Bool, Bool)}, the type of 2-tuples of two Boolean values, has a total of $2 \cdot 2 = 4$ 
              possible values.
    \end{notelist}
    \item An example:
    \begin{lstlisting}
    data DimensionalValue =
        DimensionalValue Float Dimension
    \end{lstlisting}
    \begin{notelist}
        \item \code{DimensionalValue} represents a \code{Float} value tagged with a unit of measure
              of type \code{Dimension}. We will see later how we might describe that type.
        \item The \code{data} keyword introduces a data type definition.
        \item The identifier before the \code{=} is the name of the new type.
        \item After the \code{=}, is the types constructor definition. The first identifier is the
              \textit{data constructor}, followed by arbitrarily many field declarations. 
        \item Our \code{DimensionalValue} type is equivalent to a tuple \code{(Float, Dimension)},
              but we have given it a distinct and descriptive name.
    \end{notelist}
    
    \item \textit{Sum types}: A data type with one or more alternatives.
    \begin{notelist}
        \item Enumerations are the archetypal sum types
        \item An example:
        \begin{lstlisting}
        data Dimension = Seconds
                       | Meters
                       | Newtons
        \end{lstlisting}
        \begin{notelist}
            \item As before \code{data} introduces a new type, here named \code{Dimension}.
            \item The vertical pipe \code{|} separates the various alternatives.
            \item Each alternative is given as a constructor definition as described above.
                  Here we define three simple type constructors, \code{Seconds}, \code{Meters}, and
                  \code{Newtons}.
            \item These identifiers can be used as literal values of the type \code{Dimension}. 
        \end{notelist}
        \item The ``size'' of a sum type is the sum of the sizes of the types of its alternatives.
        \item The type \code{Dimension} has $1 + 1 + 1 = 3$ possible values.
        \item Haskell's \code{Bool} data type is defined as a sum type with data constructors \code{True} and \code{False}.
    \end{notelist}

    \item The power of algebraic data types comes when we combine the two: sums of products and products of sums:
    \begin{lstlisting}
    data PlaneTicket
        = PlaneTicket Section MealOption

    data Section | Coach
                 | Business
                 | FirstClass

    data MealOption = Regular
                    | Vegetarian

    data TravelDetails = Train
                       | Automobile
                       | Plane PlaneTicket
    \end{lstlisting}
    \begin{notelist}
        \item Here we define several types that might describe the domain model of a travel agency application.
        \item \code{PlaneTicket} is a product type over two sum types: the section (\code{FirstClass} or \code{Coach}) and the meal option,
              (\code{Regular} or \code{Vegetarian}).
        \item \code{TravelDetails} is a sum type over two singleton data constructors \code{Train} and \code{Automobile} and a unary
              product alternative that tags \code{PlaneTicket} details with the data constructor \code{Plane}
        \item How many possible values are there for the \code{TravelDetails} type? $1 + 1 + (3 \cdot 2) = 8$.
    \end{notelist}
\end{notelist}
\end{notelist}

\section{Polymorphic types}

\begin{notelist}
    \item Earlier, we described the list and tuple types in terms of other, unspecified data types:
    \begin{notelist}
        \item \code{[a]} is the type of lists with elements of some type \code{a}.
        \item \code{(a, b)} is the type of 2-tuples with first element of some type \code{a} and second 
              element of some type \code{b}.
    \end{notelist}
    \item Here, \code{a} and \code{b} are \textit{type variables}.
    \item Lexically, type variables must begin with a lowercase letter. Concrete data types (in addition to
          data constructors) must begin with an uppercase letter.
    \item Data types that contain type variables are called \textit{polymorphic types}. 
    \item This type of polymorphism is known as \textit{parametric polymorphism}: substituting the concrete
          type \code{Char} for the \textit{type parameter} \code{a} in \code{[a]} gives the concrete type
          \code{[Char]}.
    \item Parametric polymorphism is distinct from the \textit{inclusion polymorphism} seen in object-oriented
          programming.
    \item This example shows how we might implement our own \code{cons}-list and 2-tuple types:
    \begin{lstlisting}
    data List a = Nil
                | Cons a (List a) 

    data Pair a b = Pair a b

    data OtherPair a = OtherPair a a
    \end{lstlisting}
    \begin{notelist}
        \item Introducing type variables on the left-hand side of the \code{=} indicates that we are
              defining a polymorphic types. \code{List} is parametric in a single type variable \code{a}
              and \code{Pair} is parametric in two type variables, \code{a} and \code{b}. 
        \item The two type variables called \code{a} in the definitions of \code{List} and \code{Pair} are
              distinct.
        \item What is the difference between our definition of \code{Pair} and \code{OtherPair}? \code{OtherPair}
              is parametric in only one type variable so both of its elements must be of the same type.
        \item We see that \code{List} is a a ``sum of products'': A \code{List} of \code{a}s is either
              the empty list \code{Nil} \textit{or} it is a value of type \code{a} followed by another
              \code{List} of \code{a}s. Thus, \code{List a} is a recursively-defined data type.
        \item Let us also make a distinction here between:
        \begin{notelist}
            \item a \textit{concrete type}, like \code{List Integer} or \code{(String, Dimension)} that has
                  no type variables;
            \item a \textit{polymorphic type} like \code{List a} that has one or more type variables;
            \item a \textit{type constructor} like \code{List} that, if ``applied'' to a concrete type,
                  yields concrete type, and if ``applied'' to a type variable yields a polymorphic type.
            \begin{notelist}
                \item Type constructors are distinct from, but analagous to, data constructors.
                \item A data constructor with fields, when applied to values to populate those fields, yields a
                      value of the type associated with that data constructor.
                \item A type constructor that admits type variables, when applied to types to instantiate those
                      type variables, yields an instantiation of the associated polymorphic type.
            \end{notelist}
        \end{notelist}
    \end{notelist}
\end{notelist}

\section{Function Types}

\begin{notelist}
    \item The examples we have looked at so far are for the types of values. However, Haskell supports
          \textit{first-class functions}: functions can be passed as parameters into functions and be
          be returned as the result of a function.
    \item That is to say, in Haskell, functions \textit{are} values. So how do we describe their types?
    \item First, we never actually define new function types with \code{data}, although we can define
          synonyms for function types with \code{type}.
    \item The one true function type constructor is \code{->}, as in \code{a -> b}, the polymorphic
          type of functions with domain \code{a} and co-domain \code{b}.
    \begin{notelist}
        \item What does the function type \code{a -> a} represent? Functions with identical domain and co-domain.
        \item With no other information about the type \code{a}, what sort of function can have
              the type \code{a -> a}? The identity function.
    \end{notelist}
    \item The functions described by \code{->} appear to only have one parameter, the type on the 
          left of the \code{->}. Haskell has operations (read: functions) like addition that take two parameters,
          so how can we describe the type of such a function?
    \item Recall that functions can return other functions as their result. Haskell models multi-parameter
          functions with single parameter functions that return a new function ready to consume more parameters.
          This technique is called \textit{currying}, named for the logician Haskell Curry.
    \begin{lstlisting} 
    add x = \y -> x + y
    \end{lstlisting} 
    \begin{notelist}
        \item We define \code{add} as a function that takes a single parameter \code{x}.
        \item It returns an anonymous function, introduced by \code{\\} (meant to suggest the Greek $\lambda$).
              It's parameter is called \code{y}. The result of this anonymous function is the sum of \code{x + y}. 
        \item When calling \code{add}, the actual parameter provided for the formal parameter \code{x} is preserved
              in a \textit{closure} that, along with the body of the anonymous function, makes up the
              function value we return.
    \end{notelist}
    
    \item Haskell does not actually inconvenience us by requiring this notation. We can just define \code{add} as:
    \begin{lstlisting} 
    add x y = x + y
    \end{lstlisting} 

    \item However, Haskell really is using currying under the hood. As such, we can \textit{partially apply}
          functions. Even with the simple definition, \code{add 5} is not an error, it returns a function value
          ready to accept another argument and add it to 5.

    \item Now the type of \code{add} should be more clear. Assuming we are only adding \code{Integers}, it must
          be \code{Integer -> (Integer -> Integer)}. 

    \item \code{->} is right associative, so we can simplify this to just \code{Integer -> Integer -> Integer}.

    \item In this form, we can view the type after the last \code{->} as the return type of the function and
          all the other types as the types of the function's parameters.

    \item We still need parentheses for grouping if one of the parameters is a function:
    \begin{notelist}
        \item Consider the function \code{map :: (a -> b) -> [a] -> [b]}.
        \item What are the types of the parameters and return value of \code{map}? The first parameter is a function
              with domain \code{a} and co-domain{b}. The second parameter is a list of \code{a}s. The result
              is a list of \code{b}s.
        \item How is that different from \code{map' :: a -> b -> [a] -> [b]}? \code{map'} takes three parameters
              (an \code{a}, a \code{b}, and a list of \code{a}s) and returns a list of \code{b}s.
        \item To what extent can you infer the semantics of \code{map} \textit{from its type alone}?
    \end{notelist}
    
    \item In general, we call functions that have one or more functions as their parameters or that return functions
          as their result \textit{higher-order functions}. As we will see, they central to more advanced techniques
          in functional programming.
\end{notelist}
