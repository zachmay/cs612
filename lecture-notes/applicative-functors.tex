\section{Applicative Functors}

When we looked at the \code{Functor} type class, we saw in \code{fmap} a way for us to lift
normal functions into the domain of computational contexts, the \code{Functor} instances, where
they can operate on values in those contexts.

But recall that in Haskell, functions are themselves first-class values. So how do we use a function
that is itself in a computational context? This question is answered by the concept of applicative
functors, realized in Haskell with the \code{Applicative} type class.

Consider this scenario. We are given an \code{Integer} with the possibility that it might not
actually be there, i.e., \code{Maybe Integer}. We are also given a function to apply to that value,
again with the possibility that it might not actually be there, i.e., \code{Maybe (Integer -> Integer)}.
Of course, since either the function or the parameter might be \code{Nothing}, the return value needs
to be able to propagate this possibility of failure. How would we accomplish that?

\begin{lstlisting}
maybeApplyToMaybe :: Maybe (Integer -> Integer) -> Maybe Integer -> Maybe Integer
maybeApplyToMaybe Nothing _ = Nothing
maybeApplyToMaybe _ Nothing = Nothing
maybeApplyToMaybe (Just f) (Just x) = Just \$ f x
\end{lstlisting}

There are three cases:

\begin{notelist}
    \item Line 2: When we get no function to apply, the argument doesn't matter: the result is \code{Nothing}.
    \item Line 3: When we get no argument to apply the function to, the function doesn't matter: the result is \code{Nothing}.
    \item Line 4: When we actually get a function and an argument, we can unwrap them from their \code{Maybe} wrapper,
          apply the function to the argument, and return the result wrapped back up.
\end{notelist}

What happens if we want to use a function with two or more arguments in this way? We don't actually have to write any more
code: Haskell's default of curried functions gives us native partial application of functions and \code{maybeApplyToMaybe}
gets that for free.

Say we have a function \code{add :: Integer -> Integer -> Integer}. The expression \code{add 5} has type \code{Integer -> Integer}.

If we look back at line 4 of the previous example, we apply our unwrapped function to the unwrapped argument, and return the
result wrapped back up. Since the partial application \code{add 5} returns a function of type \code{Integer -> Integer}, 
\code{maybeApplyToMaybe (Just add) (Just 5)} returns a value of type \code{Maybe (Integer -> Integer)}. 

\subsection{The \code{Applicative} Type Class}

We saw in the previous section that \code{map} for lists and \code{applyToMaybe} shared a common pattern, which led us to the
general \code{Functor} type class. The same idea works here, and the resulting type class is called \code{Applicative}, short for
\textit{applicative functor}.

\begin{lstlisting}
class Functor f => Applicative f where
    pure  :: a -> f a
    (<*>) :: f (a -> b) -> f a -> f b
\end{lstlisting}

The \code{Applicative} type class has its own class constraint: every instance of \code{f} of \code{Applicative} must also
be an instance of \code{Functor}. In fact, most of the standard library's \code{Functor} instances are also \code{Applicative}s
as well.

Importantly, though, \code{Applicative} and its functions are not included in the Prelude and must be imported manually from
the \code{Control.Applicative} module.

\subsection{\code{Applicative}'s Functions}

If we look back at \code{maybeApplyToMaybe :: Maybe (Integer -> Integer) -> Maybe Integer -> Maybe Integer}, we see simply
a monomorphic instance of the more general, polymorphic type of \code{(<*>)}. In fact, Haskell would have inferred a more
general type for \code{maybeApplyToMaybe}: \code{Maybe (a -> b) -> Maybe a -> Maybe b} and our implementation is essentially 
the standard library's implementation of \code{(<*>)} for \code{Maybe}.

That means we can scrap our 17 character function name and rewrite \code{maybeApplyToMaybe (Just (+5)) (Just 2)} as 
\code{Just (+5) <*> Just 2}.

In general, \code{(<*>)} is the function (used infix like an operator) that takes a function in some \code{Applicative} context
\code{f} and a value in the same context and handles the plumbing of unwrapping the function and the value, applying the function
to the value, and returning the result wrapped back up in the \code{f} context.

We haven't said much about the other half of the \code{Applicative} class, but it's quite simple and the type is very telling.
What makes sense for the type \code{a -> f a}?

We're getting a value of any type and returning a value of that type in the context described by \code{f}. Without knowing
anything about the type \code{a} of the argument, we can't modify it in any way. So, from the type alone, we can surmise that
\code{pure} probably injects a value into the context described by \code{f} in some default way.

What might \code{Maybe}'s implementation of \code{pure} look like? It's literally just \code{Just}!

\subsection{Building a Better \code{fmap}}

It might seem that all we really get from \code{Applicative} is a way to factor out the unwrapping required to apply a function
in a context to a value in a context, but there is more here.

Consider the operator \code{(<\$>) :: (a -> b) -> f a -> f b} provided by the \code{Control.Applicative} module.
It that takes a function, injects it into \code{f} with \code{pure}, and then uses \code{(<*>)} to apply it to a value
in \code{f}. 

The type of \code{(<\$>)} should look familiar: it's the same type as \code{fmap}. In fact, assuming both \code{Functor}
and \code{Applicative} laws (which we will see in a moment) the two are synonyms: \code{g <\$> pure x == fmap g \$ pure x}.

What we really get out of \code{Applicative} is a \textit{better} version of \code{fmap}.

Suppose we called \code{fmap (+) (Just 3)}. This isn't an error, we will just partially apply \code{(+)} to the wrapped
value \code{3} and get back basically \code{Just (3+)}, a function value inside a \code{Maybe} context, exactly where 
we were at the beginning of this section.

Now that we have seen how \code{Applicative}s work, we have the tools needed to finish \code{fmap}ping a function 
with two parameters: \code{fmap (+) (Just 3) <*> Just 5}, in \code{Applicative} terms: \code{pure (+) <*> Just 3 <*> Just 5},
or even more idiomatically: \code{(+) <\$> Just 3 <*> Just 5}.

From a practical perspective, we could write a function that attempted to build a \code{PlaneTicket} value based on
optional \code{Section} and \code{MealOption} values:

\begin{lstlisting}
createPlaneTicket
    :: Maybe Section
    -> Maybe MealOption
    -> Maybe PlaneTicket
createPlaneTicket section meal
    = PlaneTicket <\$> section <*> meal
\end{lstlisting}

\code{createPlaneTicket} uses the language of \code{Applicative} to succinctly lifts the \code{PlaneTicket} data constructor
into the \code{Maybe} context where the \code{MealOption}s and \code{Section}s might not exist.

In fact, the type that Haskell would actually infer for \code{createPlaneTicket} is
\code{(Applicative f) => f Section -> f MealOption -> f PlaneTicket} and would work for any \code{Applicative}, such as 
lists, which we will take a look at shortly.

\subsection{\code{Applicative} Laws}

There are four laws that \code{Applicative} instances should follow, with the same motivations and caveats described when
we discussed the \code{Functor} laws.

\begin{notelist}
    \item \textbf{Identity}: \code{pure id <*> v == v}
    \item \textbf{Homomorphism}: \code{pure g <*> pure x == pure (g x)}
    \item \textbf{Interchange}: \code{g <*> pure x == pure (\$ x) <*> g}
    \item \textbf{Composition}: \code{g <*> (h <*> k) == pure (.) <*> g <*> h <*> k}
    \item \textbf{Functor Instance}: \code{fmap g x == pure g <*> x}
\end{notelist}

The \textit{identity law} can be thought of as putting an upper bound on what can actually happen inside the plumbing of
the \code{Applicative} implementation. If that plumbing does anything that fails to preserve identity, it
is not a proper \code{Applicative}.

The intuition behind the \textit{homomorphism law} is that these operations are just lifting function application into
\code{Applicative} contexts. If we have a function \code{f} and a value \code{x}, inject each into the context
via \code{pure} and ``apply'' them via \code{(<*>)}, we should get the same thing as if we had injected
\code{f x} into the context directly.

The \textit{interchange law} is a bit tricky. To start, remember that the \code{(\$)} operator is just function application
with a very low precedence. So \code{(\$ y)} is a function that takes a function and applies it to \code{y}.
What the interchange law is trying to express is that the order in which we evaluate the function and its 
parameter should not matter in a proper \code{Applicative} instance.

We can think of the \textit{composition law} as formalizing an associative property for \code(<*>) in terms 
of Haskell's standard function composition operator \code{(.)}.

Finally, the \textit{functor instance law} describes how an \code{Applicative} instance should behave relative to
its \code{Functor} instance and is required for the equivalence between \code{(<\$>)} and \code{fmap} to hold for
an \code{Applicative} instance.

\cite{typeclassopedia}[section 4.2] offers a bit more detail on the \code{Applicative} laws.

\subsection{\code{Applicative} and Lists}

When we looked at \code{Functor}s, our two canonical examples were \code{Maybe} and lists, but we haven't really mentioned
lists yet in this section. The problem isn't that lists aren't \code{Applicative}s, it's that there are two perfectly
reasonable ways to implement the \code{Applicative} instance for lists!

Lists are a context that support zero or more values. So suppose we had a list of functions and wanted to apply them
(in the \code{Applicative} sense) to some values also in a list context: 

\begin{lstlisting}
[(+1), (*2), (^3)] <*> [4, 5, 6]
\end{lstlisting}

We could certainly interpret this as pair-wise application, applying \code{(+1)} to \code{4}, \code{(*2)} to \code{5}, etc.

However, recall that we could view lists not just as a container of zero or more values but as a kind of non-deterministic value
where \code{[4, 5, 6]} represents a value that might be any one of those numbers. In this interpretation, it might make more
sense to do apply each function from the left-hand list to each value value in the right-hand list.

The result is a list containing the possible values when a non-deterministic function is applied to a non-deterministic
value, yielding a total of 9 possible values in this example.

In fact, the Haskell library's \code{Applicative} instance for lists uses the latter interpretation. The implementation looks like this:

\begin{lstlisting}
instance Applicative [] where
    pure x    = [x]
    gs <*> xs = [ g x | g <- gs, x <- xs ]
\end{lstlisting}

\begin{notelist}
    \item \code{pure} injects a value into the list context by creating a singleton list containing that value.
    \item \code{(<*>)} applies each function from the left-hand list to each value in the right-hand list as discussed. It does
          so via Haskell's \textit{list comprehension} syntax.
\end{notelist}

What about the pair-wise version of \code{Applicative} for lists? Due to language constraints, the list type can't have
two implementations for the same type class. Instead, Haskell offers type called \code{ZipList} that wraps a normal
list but offers a different \code{Applicative} instance:

\begin{lstlisting}
newtype ZipList a = ZipList { getZipList :: [a] }
 
instance Applicative ZipList where
   pure x = repeat x
   (ZipList gs) <*> (ZipList xs) = ZipList (zipWith (\$) gs xs)
\end{lstlisting}

\begin{notelist}
    \item The \code{newtype} keyword defines a type synonym that is checked at compile time but discarded so there is no
          run-time overhead. A \code{ZipList} can never be used as a normal list, but there's no additional overhead.
          Record syntax is used here to automatically create a function \code{getZipList} to translate normal lists to
          \code{ZipList}s.
    
    \item \code{zipWith} takes two lists and applies a function pair-wise to the elements of those lists. In this case
          the function is \code{(\$)}, which we have seen previously. So the \code{ZipList} instance of \code{Applicative}
          is implementing pair-wise application.

    \item Because \code{zipWith} truncates the result to the length of the shorter of its two arguments, it makes sense
          for \code{pure} to inject values into the \code{ZipList} context by creating an infinite list via \code{repeat}.

    \item If we used the same \code{pure} implementation as normal lists, \code{pure g <*> [1..] == [g 1]} and the functor
          instance law no longer holds.
\end{notelist}

\subsection{Summary}

In this section we have looked at applicative functors and Haskell's \code{Applicative} type class. We have seen how
\code{Applicative} offers an abstraction for applying functions even when the functions themselves were wrapped up in
a context just as \code{fmap} allowed us to apply bare functions to values in a context.

In the next section we will discuss the \code{Monad} type class and see how it further extends the notion of computational
contexts that we have built up via \code{Functor} and \code{Applicative}.

