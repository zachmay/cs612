\documentclass[12pt]{article}
% \usepackage{palatino}
\usepackage{epsfig}
\usepackage{epstopdf} % for \psfig{.eps}
\usepackage{amssymb} % for \varnothing
\usepackage{amsmath} % for \pmod
\usepackage{url} % to split long names
\usepackage{color}
\usepackage{listings}
\usepackage{hyperref}
\setlength{\topmargin}{-0.1in}
\setlength{\textheight}{8.0in}
% \newcommand\note[1]{{\color{red} #1}}
\pagestyle{myheadings}
\markboth{Haskell Lecture Notes}{Haskell Lecture Notes}
\newcommand\note[1]{\raggedright\fbox{#1}}
% \newcommand\ind{\hspace*{.3in}}
\newcommand\code[1]{\texttt{\textbf{#1}}}
\def\>{\hspace*{0.2in}}
\newenvironment{notelist}{\begin{list}
   {$\bullet$}
   {\setlength{\itemsep}{0in}}}
   {\end{list}}
    
% Options for code listings
\lstset{
    frame=single,
    language=Haskell,
    numbers=left
}

\author{Zachary May}
\title{Haskell Lecture Notes}
\begin{document}
\maketitle

\section{Introducing Haskell}
\begin{notelist}
\item Haskell is a \textbf{statically-typed}, \textbf{non-strict}, \textbf{pure} \textbf{functional}
      programming language.
	\begin{notelist}
	\item \textbf{Functional:} Conceptually, computation proceeds via the application of functions to parameters rather than by 
          sequential instructions manipulating values in memory. Functions are first-class values.
	\item \textbf{Pure:} The functions in question are more like mathematical functions than procedures. They map values in an input domain
          to values in an output domain. Pure functions have no side effects. These features are valuable when reasoning about
          and testing our code.
	\item \textbf{Non-strict:} By default, Haskell uses a lazy evaluation strategy. Expressions do not need to be evaluated until the 
          results are needed. For example, Haskell can cleanly represent infinite lists because the language only evaluates such
          an expression as needed.
    \item \textbf{Statically-typed:} The type of every expression is known at compile time, preventing run-time errors caused by type
          incompatibilities. This feature prevents things like Java's \code{NullPointerException}, because a function that claims it returns
          a value of a specific type must live up to that promise. Returning the Haskell equivalent of \code{null} is a compile-time
          type error. Additionally, Haskell makes use of a technique called \textbf{type inference} to figure out the types of most things
          without needing explicit type annotations.
	\end{notelist}
\end{notelist}
\pagebreak

\section{A Sample Program}

\begin{lstlisting}
sumSquares :: Integer -> [Integer] -> Integer
sumSquares count numbers =
    sum (take count (map (^2) numbers))

printSquares :: IO ()
printSquares =
   print $ sumSquares 10 [1..]
\end{lstlisting}

\begin{notelist}
    \item The type signature in line 1 describes a function with two parameters: an \code{Integer} and a list
          of \code{Integer}s. It returns an \code{Integer}. In general, the type after the last \code{->}
          is the return value, and the others are the parameters.
    \item Haskell would actually infer a more general type than what we see in the type annotation 
          on line 1. Haskell programmers usually include type annotations as a form of machine-checked
          documentation.
    \item Haskell functions are defined with an equational syntax as seen in lines 2-3: \code{sumSquares}
          applied to the parameters \code{count} and \code{numbers} is equal to the expression on the
          right-hand side of the equation.
    \item Line 3 shows us several examples of \textbf{function application}. The notation for function application
          is lightweight: simple juxtaposition of terms. The parentheses here are for grouping only. 
    \item Working from the inside out, \code{map} applies a function to each element in a list, producing a new 
          new list containing the resulting values.
    \item \code{(\string^2)} is called a \textbf{section}, a shorthand for an anonymous function whose parameters
          fills in the blank for a binary operator, exponentiation in this case. So \code{map (\string^2)}
          transforms a list of numbers into their squares.
    \item Given a number $n$ and a list, \code{take} returns the first $n$ items of the list, or the whole list
          if it has fewer than $n$ elements. \code{sum} takes a list of numbers and returns the sum.
    \item Lines 5-7 define another function, \code{printSquares}. It takes no parameters, and its return type, \code{IO ()},
          is quite interesting.
    \item Look back at the type \code{[Integer]}. The list type \code{[]} is a
          \textbf{parameterized} type. That is, list itself is not a concrete type, but a \textbf{type constructor}.
          We need another type, the parameter, to create a concrete type. Similarly, \code{IO} is a type constructor.
    \item Here we instantiate \code{IO} with the type \code{()}, the \textbf{unit type} with only a single value, also
          written as \code{()}. The type \code{IO ()} represents an \textbf{I/O action} with no interesting return
          value. The \code{IO} type constructor is the technique Haskell uses to perform I/O, inherently impure,
          in a world of pure functions, via a far more general abstraction called the \textbf{monad}.
    \item Consider the expression \code{[1..]}. This expression represents the infinite list of positive
          integers. However, because Haskell evaluates expressions lazily, merely describing the value is not enough to
          force the run-time to evaluate the entire thing. In fact, in this case, evaluation of \code{[1..]} is only
          forced by \code{print} asking for the value produced by \code{sum}, which demands the values yielded by
          \code{take count}, and so on, on demand.
    \item Notice the \code{\$} in line 7. Pronounced ``applied to'', it is an operator with extremely low
          precedence that breaks up the very high precedence of function application, allowing us to avoid nesting
          parentheses. We could have defined \code{sumSquares} using \code{\$} as:
          \code{sum \$ take count \$ map (\string^2) numbers}. 
\end{notelist}

\section{Basic Haskell Types}

In this section, we take a deeper look at Haskell's type system.

\begin{notelist}
\item Haskell offers the some basic primitive data types we would expect: \cite{haskell98}
\begin{notelist}
    \item \code{Int} and \code{Integer}: machine-sized and arbitrary precision integers, respectively.
    \item \code{Float} and \code{Double}: single- and double-precision floating point numbers.
    \item \code{Char}: Single Unicode characters.
    \item \code{Bool}: Boolean values (but see below, \code{Bool} is actually a composite type).
\end{notelist}

\item Additionally, we have several composite data types:
\begin{notelist}
    \item \code{cons} lists: Lisp-style linked lists. The type is written \code{[a]} where \code{a}
          is another type.
    \item \code{String}: character strings; \code{String} is literally just a \textit{type synonym} for \code{[Char]}.
    \item Tuples: $k$-element tuples, $k \geq 2$. The type of an $n$-element tuple is \code{(a1, a2, ..., an)} where
          \code{a1}, \code{a2}, \ldots, and \code{an} are other types.
\end{notelist}

\item New data types are introduced with the \code{data} keyword.

\item Type synonyms can be introduced with the \code{type} keyword. \code{type EmailAddress = String} says that 
      the identifier \code{EmailAddress} can be used interchangeably with the type \code{String}. The advantage
      is that \code{EmailAddress} is more descriptive.

\item The keyword \code{newtype} creates a more controlled sort of type synonym.
\begin{notelist}
    \item If we wanted a type to describe e-mail address values but did not want it to be
          interchangeable with \code{String}s in general, we could define a new type that simply ``tags''
          a \code{String} value: \code{data EmailAddress = EmailAddress String}.
    \item This comes with some amount of overhead each time we want to ``unwrap'' the \code{EmailAddress} and
          get at the underlying \code{String}.
    \item Instead we can use \code{newtype EmailAddress = EmailAddress String}. Haskell's type checker treats
          this exactly like type introduced with \code{data}, but drops the tagging for purposes of code
          generation, eliminating the overhead required to ``unwrap'' the \code{String}.
\end{notelist}

\item Although Haskell can \textit{infer} the types of most all expressions, types can be stated explicitly
      with a type annotation using \code{::}. For example:
\begin{lstlisting}
nothing :: [String]
nothing = []

moreNothing = []
\end{lstlisting}
\begin{notelist}
    \item We define two values, \code{nothing} and \code{moreNothing}.
    \item Although the equational definitions are identical, we explicitly define the type of
          \code{nothing} to be of type \code{[String]}, a list of strings, with the type
          annotation on line 1.
    \item Without an explicit type annotation, Haskell will infer the type of \code{moreNothing},
          in this case, the more general type \code{[a]}. (This is a \textit{polymorphic type} which
          we will discuss later.)
\end{notelist}

\item Haskell's lists and tuples are specific examples of the language's \textit{algebraic type system}.
\item Algebraic data types were introduced in the Hope programming language in 1980. \cite{hope}
\item An algebraic type system generally offers two sorts of types:
\begin{notelist}
    \item \textit{Product types}: A data type with one or more fields.
    \begin{notelist}
        \item Tuples are the archetypal product type.
        \item The ``size'' of a product type is the product of the sizes of the types of its fields.
        \item E.g., \code{(Bool, Bool)}, the type of 2-tuples of two Boolean values, has a total of $2 \cdot 2 = 4$ 
              possible values.
    \end{notelist}
    \item An example:
    \begin{lstlisting}
    data DimensionalValue =
        DimensionalValue Float Dimension
    \end{lstlisting}
    \begin{notelist}
        \item \code{DimensionalValue} represents a \code{Float} value tagged with a unit of measure
              of type \code{Dimension}. We will see later how we might describe that type.
        \item The \code{data} keyword introduces a data type definition.
        \item The identifier before the \code{=} is the name of the new type.
        \item After the \code{=}, is the types constructor definition. The first identifier is the
              \textit{data constructor}, followed by arbitrarily many field declarations. 
        \item Our \code{DimensionalValue} type is equivalent to a tuple \code{(Float, Dimension)},
              but we have given it a distinct and descriptive name.
    \end{notelist}
    
    \item \textit{Sum types}: A data type with one or more alternatives.
    \begin{notelist}
        \item Enumerations are the archetypal sum types
        \item An example:
        \begin{lstlisting}
        data Dimension = Seconds
                       | Meters
                       | Newtons
        \end{lstlisting}
        \begin{notelist}
            \item As before \code{data} introduces a new type, here named \code{Dimension}.
            \item The vertical pipe \code{|} separates the various alternatives.
            \item Each alternative is given as a constructor definition as described above.
                  Here we define three simple type constructors, \code{Seconds}, \code{Meters}, and
                  \code{Newtons}.
            \item These identifiers can be used as literal values of the type \code{Dimension}. 
        \end{notelist}
        \item The ``size'' of a sum type is the sum of the sizes of the types of its alternatives.
        \item The type \code{Dimension} has $1 + 1 + 1 = 3$ possible values.
        \item Haskell's \code{Bool} data type is defined as a sum type with data constructors \code{True} and \code{False}.
    \end{notelist}

    \item The power of algebraic data types comes when we combine the two: sums of products and products of sums:
    \begin{lstlisting}
    data PlaneTicket
        = PlaneTicket Section MealOption

    data Section | Coach
                 | Business
                 | FirstClass

    data MealOption = Regular
                    | Vegetarian

    data TravelDetails = Train
                       | Automobile
                       | Plane PlaneTicket
    \end{lstlisting}
    \begin{notelist}
        \item Here we define several types that might describe the domain model of a travel agency application.
        \item \code{PlaneTicket} is a product type over two sum types: the section (\code{FirstClass} or \code{Coach}) and the meal option,
              (\code{Regular} or \code{Vegetarian}).
        \item \code{TravelDetails} is a sum type over two singleton data constructors \code{Train} and \code{Automobile} and a unary
              product alternative that tags \code{PlaneTicket} details with the data constructor \code{Plane}
        \item How many possible values are there for the \code{TravelDetails} type? $1 + 1 + (3 \cdot 2) = 8$.
    \end{notelist}
\end{notelist}
\end{notelist}

\section{Polymorphic types}

\begin{notelist}
    \item Earlier, we described the list and tuple types in terms of other, unspecified data types:
    \begin{notelist}
        \item \code{[a]} is the type of lists with elements of some type \code{a}.
        \item \code{(a, b)} is the type of 2-tuples with first element of some type \code{a} and second 
              element of some type \code{b}.
    \end{notelist}
    \item Here, \code{a} and \code{b} are \textit{type variables}.
    \item Lexically, type variables must begin with a lowercase letter. Concrete data types (in addition to
          data constructors) must begin with an uppercase letter.
    \item Data types that contain type variables are called \textit{polymorphic types}. 
    \item This type of polymorphism is known as \textit{parametric polymorphism}: substituting the concrete
          type \code{Char} for the \textit{type parameter} \code{a} in \code{[a]} gives the concrete type
          \code{[Char]}.
    \item Parametric polymorphism is distinct from the \textit{inclusion polymorphism} seen in object-oriented
          programming.
    \item This example shows how we might implement our own \code{cons}-list and 2-tuple types:
    \begin{lstlisting}
    data List a = Nil
                | Cons a (List a) 

    data Pair a b = Pair a b

    data OtherPair a = OtherPair a a
    \end{lstlisting}
    \begin{notelist}
        \item Introducing type variables on the left-hand side of the \code{=} indicates that we are
              defining a polymorphic types. \code{List} is parametric in a single type variable \code{a}
              and \code{Pair} is parametric in two type variables, \code{a} and \code{b}. 
        \item The two type variables called \code{a} in the definitions of \code{List} and \code{Pair} are
              distinct.
        \item What is the difference between our definition of \code{Pair} and \code{OtherPair}? \code{OtherPair}
              is parametric in only one type variable so both of its elements must be of the same type.
        \item We see that \code{List} is a a ``sum of products'': A \code{List} of \code{a}s is either
              the empty list \code{Nil} \textit{or} it is a value of type \code{a} followed by another
              \code{List} of \code{a}s. Thus, \code{List a} is a recursively-defined data type.
        \item Let us also make a distinction here between:
        \begin{notelist}
            \item a \textit{concrete type}, like \code{List Integer} or \code{(String, Dimension)} that has
                  no type variables;
            \item a \textit{polymorphic type} like \code{List a} that has one or more type variables;
            \item a \textit{type constructor} like \code{List} that, if ``applied'' to a concrete type,
                  yields concrete type, and if ``applied'' to a type variable yields a polymorphic type.
            \begin{notelist}
                \item Type constructors are distinct from, but analagous to, data constructors.
                \item A data constructor with fields, when applied to values to populate those fields, yields a
                      value of the type associated with that data constructor.
                \item A type constructor that admits type variables, when applied to types to instantiate those
                      type variables, yields an instantiation of the associated polymorphic type.
            \end{notelist}
        \end{notelist}
    \end{notelist}
\end{notelist}

\section{Function Types}

\begin{notelist}
    \item The examples we have looked at so far are for the types of values. However, Haskell supports
          \textit{first-class functions}: functions can be passed as parameters into functions and be
          be returned as the result of a function.
    \item That is to say, in Haskell, functions \textit{are} values. So how do we describe their types?
    \item First, we never actually define new function types with \code{data}, although we can define
          synonyms for function types with \code{type}.
    \item The one true function type constructor is \code{->}, as in \code{a -> b}, the polymorphic
          type of functions with domain \code{a} and co-domain \code{b}.
    \begin{notelist}
        \item What does the function type \code{a -> a} represent? Functions with identical domain and co-domain.
        \item With no other information about the type \code{a}, what sort of function can have
              the type \code{a -> a}? The identity function.
    \end{notelist}
    \item The functions described by \code{->} appear to only have one parameter, the type on the 
          left of the \code{->}. Haskell has operations (read: functions) like addition that take two parameters,
          so how can we describe the type of such a function?
    \item Recall that functions can return other functions as their result. Haskell models multi-parameter
          functions with single parameter functions that return a new function ready to consume more parameters.
          This technique is called \textit{currying}, named for the logician Haskell Curry.
    \begin{lstlisting} 
    add x = \y -> x + y
    \end{lstlisting} 
    \begin{notelist}
        \item We define \code{add} as a function that takes a single parameter \code{x}.
        \item It returns an anonymous function, introduced by \code{\\} (meant to suggest the Greek $\lambda$).
              It's parameter is called \code{y}. The result of this anonymous function is the sum of \code{x + y}. 
        \item When calling \code{add}, the actual parameter provided for the formal parameter \code{x} is preserved
              in a \textit{closure} that, along with the body of the anonymous function, makes up the
              function value we return.
    \end{notelist}
    
    \item Haskell does not actually inconvenience us by requiring this notation. We can just define \code{add} as:
    \begin{lstlisting} 
    add x y = x + y
    \end{lstlisting} 

    \item However, Haskell really is using currying under the hood. As such, we can \textit{partially apply}
          functions. Even with the simple definition, \code{add 5} is not an error, it returns a function value
          ready to accept another argument and add it to 5.

    \item Now the type of \code{add} should be more clear. Assuming we are only adding \code{Integers}, it must
          be \code{Integer -> (Integer -> Integer)}. 

    \item \code{->} is right associative, so we can simplify this to just \code{Integer -> Integer -> Integer}.

    \item In this form, we can view the type after the last \code{->} as the return type of the function and
          all the other types as the types of the function's parameters.

    \item We still need parentheses for grouping if one of the parameters is a function:
    \begin{notelist}
        \item Consider the function \code{map :: (a -> b) -> [a] -> [b]}.
        \item What are the types of the parameters and return value of \code{map}? The first parameter is a function
              with domain \code{a} and co-domain{b}. The second parameter is a list of \code{a}s. The result
              is a list of \code{b}s.
        \item How is that different from \code{map' :: a -> b -> [a] -> [b]}? \code{map'} takes three parameters
              (an \code{a}, a \code{b}, and a list of \code{a}s) and returns a list of \code{b}s.
        \item To what extent can you infer the semantics of \code{map} \textit{from its type alone}?
    \end{notelist}
    
    \item In general, we call functions that have one or more functions as their parameters or that return functions
          as their result \textit{higher-order functions}. As we will see, they central to more advanced techniques
          in functional programming.
\end{notelist}

\section{Ad-Hoc Polymorphism with Typeclasses}

\begin{notelist}
    \item At the machine level, adding two integers is quite a different operation from
          adding two floating-point numbers. High-level languages, in an effort to hide
          this detail, \textit{overload} the semantics of the addition operator to support 
          these distinct operations using the same operator. 

    \item In a strongly-typed language like Haskell, what might the type of \code{(+)} be?
    \begin{notelist}
        \item In Haskell, infix binary operators are just syntactic sugar for functions
              of two arguments. When referring to binary operators outside of their
              normal infix notation, Haskell requires them to be surrounded by parentheses.
    \end{notelist}

    \item \code{Int -> Int -> Int} or similar is insufficient, since the type of \code{(+)}
          needs to be general enough to describe adding together two operands
          of many different numeric types.

    \item \code{a -> a -> a} seems promising: two operands and a result, all of the same type. However,
          this type signature \textit{unifies} with \code{TravelDetails -> TravelDetails -> TravelDetails}
          and addition of that type does not make sense.

    \item What happens if we ask the Haskell compiler to infer the type of \code{(+)}?
    \begin{notelist}
        \item The most popular Haskell compiler, GHC, has a REPL interface called GHCi.
        \item The command \code{:t \textit{expression}} asks GHCi to infer the type of
              \code{\textit{expression}}.
        \item \code{:t (+)} yields the inferred type: \code{(Num a) => a -> a -> a}.
    \end{notelist}

    \item \code{(Num a) => ...} is a \textit{class constraint} on the type signature that
          follows the \code{=>}. 

    \item Normally, a free type variable in a type signature can be unified with any type at all.
    
    \item A class constraint on a type variable restricts the types that it can be unified with
          to types that are \textit{instances} of the named \textit{type class}.

    \item So \code{(+) :: (Num a) => a -> a -> a} says that \code{(+)} is a function of
          two operands and result all of some type \code{a} \textit{where \code{a} is
          an instance of the type class \code{Num}}.

    \item A type class acts a bit like a \textit{interface} in object-oriented programming.
          It acts as a contract: any type that is an instance of a type class must implement 
          certain methods to qualify.

    \item The terminology may be a bit confusing:
    \begin{notelist}
        \item In an OO language, an object is an \textit{instance} of a \textit{class} which
              might \textit{implement} an \textit{interface}.
        \item In Haskell, a value \textit{has} a \textit{type} which might be an \textit{instance}
              of a \textit{type class}.
    \end{notelist}
\end{notelist}

\section{Basic Typeclasses from The Haskell Prelude}

Haskell offers quite a bit of functionality in its standard library. In particular, the \textit{Prelude},
the set of type and function definitions imported automatically into every program, defines 

One of the most basic typeclasses is \code{Eq}, consisting of types that implement an equality-testing
operation. Here is how it is defined:

\begin{lstlisting}
  class Eq a where
        (==), (/=)  ::  a -> a -> Bool

        x /= y  = not (x == y)
        x == y  = not (x /= y)
\end{lstlisting}

\begin{notelist}
    \item The \code{class} keyword introduces a type class definition, followed by the
          name of the type class and a type variable that we will use in the description
          of the type class's interface. Think of this type variable as a formal parameter
          in a function definition. 

    \item The \code{Eq} type class defines two required operations, equality and inequality.
          In this case, the two have exactly the same signature, so the type annotation
          is shared.

    \item If we asked Haskell to infer the type of \code{(==)}, what would we get?
          \code{(==) :: (Eq a) => a -> a -> Bool}.

    \item Then we see two equational function definitions. These are default implementations
          for \code{Eq}'s operations.

    \item This means we do not have to define both \code{(==)} and \code{(/=)}. Each is defined
          in terms of the other, so an implementation for one is enough. The compiler will
          complain if neither is implemented.

    \item Because we can define default implementations, type classes are actually more like
          \textit{abstract classes} in OO languages.
\end{notelist}

Typeclasses can themselves have class constraints. Here is the definition of \code{Ord}, which describes
operations available for totally ordered data types:

\begin{lstlisting}
class Eq a => Ord a where
    compare              :: a -> a -> Ordering
    (<), (>=), (>), (<=) :: a -> a -> Bool
    max                  :: a -> a -> a
    min                  :: a -> a -> a

data Ordering = LT | EQ | GT
\end{lstlisting}

\begin{notelist}
    \item Class constraints in a type annotation, as in line 1 above, require that the constrained
          type variable be an instance of the given typeclass.
    
    \item In this case, for a type to be an instance of \code{Ord}, it must also be an instance of
          \code{Eq}. It should be pretty clear why that is necessary.
    
    \item The full definition of \code{Ord} gives default implementations for all these operations so
          that an instance need only implement either \code{compare} or \code{(<=)}.
\end{notelist}

There are several other typeclasses worth mentioning:

\begin{notelist}
    \item \code{Show} instances can be turned into a \code{String} representation with \code{show :: (Show a) => a -> String}.
    \item \code{Read} instances know how to undo the process and turn a string into a value.
    \item \code{Bounded} instances are types with a smallest and largest values, given as two polymorphic 
          constants \code{minBound, maxBound :: (Bounded a) => a}.
    \item \code{Enum} instances are sequentially ordered types. Given a value in that sequence, we can use
          \code{succ, pred :: (Enum a) => a -> a} to get the next or previous value. We can use 
          \code{enumFromTo :: (Enum a) => a -> a -> [a]} to get a list containing the elements in the sequence
          between a start value and an end value, inclusive.
\end{notelist}

Haskell's standard library also defines a hierarchy of numeric typeclasses.

\begin{notelist}
    \item We saw that GHC would infer the type \code{(Num a) => a -> a -> a} for the \code{(+)} operator.
          That, along with \code{(*)}, \code{(-)} (the binary subtraction operator), \code{negate} (for unary
          negation), and a couple of others define the most basic interface for numeric types.
    \item \code{Fractional} extends \code{Num} with division in \code{(/)} and reciprocation in \code{recip}.
    \item \code{Floating} extends \code{Fractional} with real-valued logarithms, exponentiation, trigonometric
          functions and even \code{(Floating a) => pi :: a}, the polymorphic constant $\pi$.
\end{notelist}

The full numeric hierarchy is even richer and there is plenty of detail in the Prelude's typeclasses that we
have glossed over. Full details are available in \cite[section 6.4]{haskell98}.

\section{Creating New Typeclass Instances}

Haskell typeclasses are \textit{open}, meaning that we can define new instances of typeclasses defined in the
Prelude or elsewhere.

Let's see how we can implement some of the Prelude's basic typeclasses for a simple type.

\begin{lstlisting}
data Section = Coach
             | Business
             | FirstClass

instance Eq Section where
    FirstClass == FirstClass = True
    Business   == Business   = True
    Coach      == Coach      = True
    _          == _          = False

instance Ord Section where
    x          <= y          | x == y = True
    Coach      <= _                   = True
    Business   <= Coach               = False
    Business   <= FirstClass          = True
    FirstClass <= _                   = False
    
instance Show Section where
    show Coach      = "Coach"
    show Business   = "Business"
    show FirstClass = "FirstClass"
\end{lstlisting}

\begin{notelist}
    \item An instance declaration begins with the keyword \code{instance}, followed by equational
          definitions for the various functions defined for the class.
    \item The definition of \code{(==)} for \code{Eq} is straightforward. We define the function
          in four cases. In the first three equations, we enumerate the cases where values could
          be considered equal and the last equation is a catch-all: the underscore character 
          matches any value, so any case not matched by the first three equations will get caught
          by the fourth and will return \code{False}.
    \item We define an \code{Ord} instance by enumerating the ways in which \code{Section} values
          can be ordered. The first equation uses a \textit{guard}: \code{x} and \code{y} will match
          any values, but the match is only successful if the \textit{guard expression} evaluates
          to \code{True}.
    \item The \code{Show} instance is trivial: we simpliy define a string value to return for each
          of \code{Section}'s three data constructors.
\end{notelist}

We can imagine that defining instances for these typeclasses would be quite similar for any algebraic
data type. It seems trivial to automatically construct an \code{Eq} instance for any simple sum type.
Furthermore, because of the recursive nature of algebraic data types, it would be easy to extend that
idea to arbitrary sum-of-products types. 

\begin{lstlisting}
data S = P1 T1_1 ... T1_K1
       | P2 T2_1 ... T2_K2
         ...
       | PN TN_1 ... TN_KN

instance Eq S where
    P1 u1_1 ... u1_k1 == P1 v1_1 ... v1_k1 = u1_1 == v1_1 && .. && u1_k1 == v1_k1
    P2 u2_1 ... u2_k2 == P2 v2_1 ... v2_k2 = u2_1 == v1_1 && .. && u2_k2 == v2_k2
    ..
    PN un_1 ... uN_kN == PN vN_1 ... v1_kN = uN_1 == vN_N && .. && uN_kN == vN_kN
    _ == _ = False
\end{lstlisting}

\begin{notelist}
    \item This is pseudocode for the general form of an \code{Eq} instance for a sum of
          $N$ constructors that are each a product of $K_N$ values.
    \item In words, two \code{S} values are equal if their data constructors are equal and
          each pair of constituent values are equal.
\end{notelist}

In fact, Haskell offers the ability to \textit{derive} typeclass instances, and not just for 
\code{Eq}.

\begin{lstlisting}
data Section = Coach
             | Business
             | FirstClass
             derving (Eq, Ord, Show)
\end{lstlisting}

\begin{notelist}
    \item The \code{deriving} keyword instructs the compiler to automatically derive instances
          for the typeclasses that follow.
    \item The Haskell 98 standard can derive instances for \code{Eq}, \code{Ord}, \code{Enum},
          \code{Bounded}, \code{Show}, and \code{Read}.
    \item In automatically derived instances of \code{Ord}, \code{Enum}, and \code{Bounded},
          the order of declaration of the data constructors is used. So our derived
          \code{Ord} instance for \code{Section} still returns \code{True} for  \code{Coach <= FirstClass}.
\end{notelist}

Because the derived definitions are recursive, we might not always be able to derive instances
when the constituents of product types do not support the operations we need:

\begin{lstlisting}
data Section = Coach
             | Business
             | FirstClass BeverageOption
             derving (Eq, Ord, Show)

data BeverageOption = Wine
                    | Beer
                    | Soda
\end{lstlisting}

\begin{notelist}
    \item Here, we cannot automatically derive an \code{Eq} instance for \code{Section} because 
          \code{BeverageOption} is not an instance of \code{Eq}. We cannot determine if two
          values using the \code{FirstClass} constructor are equal because we have no way of
          checking two \code{BeverageOption} values for equality.
    \item Similarly, we could not derive an \code{Ord} instance since we have no notion of
          ordering on \code{BeverageOption}s.
    \item In this case, adding a \code{deriving} clause to our definition of the
          \code{BeverageOption} type would resolve the problem.
\end{notelist}

\section{Maybe, Lists, and The Functor Typeclass}

Now let us consider a type defined in the Haskell Prelude, \code{Maybe}:

\begin{lstlisting}
data Maybe a = Nothing
             | Just a
\end{lstlisting}

\begin{notelist}
    \item \code{Maybe} is the Haskell version of the \textbf{option type}. It offers us a way to
          represent a value that might not exist. \code{Nothing} is the null value, and
          the \code{Just} constructor wraps an actual value.
    \item For example, we might want a function that parses an integer value from a string to
          have the return type \code{Maybe Integer}, since the parse might fail.
    \item Compare this to Java's type system where \code{null} is a possible value for any
          reference type. \code{null} is a \code{Person} even though \code{null} does not respond to  
          any of \code{Person}'s methods--or any methods at all!
    \item In Haskell, on the other hand, a function that claims to return a \code{Person} always returns
          a full-fledged \code{Person} (barring exceptional failure) and a function that sometimes
          returns a \code{null}-like value must declare that in its type, e.g., \code{String -> Maybe Integer}.
\end{notelist}

The safety we get when we use an option type is nice, but it comes with some inconvenience: If I 
have a value of type \code{Maybe Integer}, how do I add five to it? In general, how do I unwrap
a value of the form \code{Just x} to get at \code{x}? It is not difficult in principle:

\begin{lstlisting}
parseInteger :: String -> Maybe Integer
# Implemented elsewhere

example1 :: String -> Integer
example1 str = case parseString str of
                  Just x  -> x + 5
                  Nothing -> 0

example2 :: String -> Maybe Integer
example2 str = case parseString str of
                  Just x  -> Just (x + 5)
                  Nothing -> Nothing
\end{lstlisting}

\begin{notelist}
    \item However, this code has some shortcomings:
    \begin{notelist}
        \item In \code{example1}, we are making an assumption about how to handle the error case
              (returning zero if the parse failed) that is now interwoven with the independent process
              of adding five.
        \item Both examples repeat the code to test both the \code{Just} and \code{Nothing} case. Moreover,
              if we used \code{Maybe} frequently (which is encouraged), this would start to get rather
              annoying.
    \end{notelist}
\end{notelist}

We will reject \code{example1} because we really would like to maintain the orthogonality of
dealing with \code{Maybe} values and our actual operation. However we can use higher-order
functions to factor out repetition we see in analyzing \code{Maybe}s.

\begin{lstlisting}
applyToMaybe :: (a -> b) -> (Maybe a) -> (Maybe b)
applyToMaybe f Nothing  = Nothing
applyToMaybe f (Just x) = Just (f x)
\end{lstlisting}

\begin{notelist}
    \item \code{applyToMaybe} factors out handling the \code{Nothing} and \code{Just} cases of \code{Maybe}
          values.
    \item We also get some vocabulary for lifting normal function application into the world of \code{Maybe}
          values.
    \item In our add five example, we can now just use \code{applyToMaybe (+5) \$ parseString str}
\end{notelist}

Here is another example. We have seen Haskell's basic, homogeneous list type. It offers us a way to
represent a collection of zero or more values of some type.

We have also seen the function \code{map :: (a -> b) -> [a] -> [b]} that applies a function
to each element in a list and returns the results collected into a new list. 

\begin{lstlisting}
map :: (a -> b) -> [a] -> [b]
map f []     = []
map f (x:xs) = f x : (map f xs)
\end{lstlisting}

If we compare \code{applyToMaybe} and \code{map}, we see some important similarities:

\begin{notelist}
    \item Both functions have an empty case and a case where one or more values are unwrapped,
          a function applied, and the result(s) wrapped back up. 
    \item If we ignore the special case of Haskell's list type syntax, the functions have analogous types
          of the form \code{(a -> b) -> f a -> f b}
\end{notelist}

\subsection{The Functor Typeclass}

In fact, this pattern is codified in Haskell with the \code{Functor} type class:

\begin{lstlisting}
class Functor f where
    fmap :: (a -> b) -> f a -> f b
\end{lstlisting}

\begin{notelist}
    \item A \code{Functor} instance is always a polymorphic data type with a single type parameter.
    \item At one level we can think of \code{Functor}s as simply mappable containers.
    \item At another level, we can think of them as values in some sort of context, where \code{fmap}
          lifts function application into that new context.
\end{notelist}

\subsubsection{Functor Instances}

Here are some instances of the \code{Functor} type class:

\begin{notelist}
    \item \textbf{\code{Maybe}}
    \begin{notelist}
        \item We can think of \code{Maybe a} as a context representing a value of type \code{a} with the 
              possibility of failure.
        \item In this context, we can think of \code{fmap} as creating new functions that know how to
              propagate these failure states.
    \end{notelist}

    \item \textbf{Lists, i.e., \code{[]}}
    \begin{notelist}
        \item The implementation of \code{fmap} for lists is literally just the Prelude's \code{map} function.
        \item From the context perspective, we can think of lists as non-deterministic values; i.e., the
              result of applying the function \code{(* 2)} to the non-deterministic value that might be
              one of \code{[1, 2, 3]} would be the non-deterministic value that might be one of
              \code{[2, 4, 6]}.
    \end{notelist}

    \item \textbf{\code{Tree}}
    \begin{notelist}
        \item Mapping over a collection makes sense for trees, but what might \code{Tree} represent from the perspective
              of values in a context?
        \item Interestingly, while mapping over elements in a set seems reasonable enough, Haskell's \code{Set} type
              cannot be directly declared an instance of \code{Functor}. Because \code{Set} is implemented via balanced
              binary trees, it has an \code{Ord} constraint on the types it can contain. This extra constraint is
              incompatible with the general \code{Functor} definition; we would need \code{fmap} to have the type
              \code{(Ord a, Ord b) => (a -> b) -> f a -> f b}.
        \item \code{Map}, however can be a \code{Functor}. Rather, maps with keys of type \code{k}, i.e., \code{Map k}
              can be a \code{Functor}. Haskell \code{Map}s are represented using balanced binary trees over the
              key type \code{k}, so there is still an \code{Ord} constraint, \code{Functor} cares about the type of
              the values, not the type of the keys.
    \end{notelist}

    \item \textbf{\code{((->) e)}}
    \begin{notelist}
        \item This type looks a big strange. Haskell's syntax does not allow it, but read this type as \code{(e ->)}.
        \item Concretely, the type of the \code{fmap} implementation here would be \code{(a -> b) -> (e -> a) -> (e -> b)}
        \item \cite{typeclassopedia} describes \code{((->) e)} as ``a (possibly infinite) set of values of \code{a}, indexed by
              values of \code{e},'' or ``a context in which a value of type e is available to be consulted in a read-only fashion.''
        \item If we have a predicate \code{isOdd :: Int -> Bool}, and \code{fmap} it over a function \code{length :: String -> Int}
              that returns the length of its parameter, we get a new function of type {String -> Bool} that returns whether or
              not the \code{String}'s length is odd.
        \item From the context perspective, \code{fmap isOdd} takes us from \code{Int}s indexed by \code{String}s to \code{Bool}s
              indexed by \code{String}s.
    \end{notelist}
\end{notelist}

\subsubsection{Functor Laws}

For the Haskell type system, anything that implements \code{fmap :: (a -> b) -> f a -> f b} is perfectly suitable
as an instance of \code{Functor}. However, the concept of functors come to us from the branch of mathematics called
category theory, where functors must satisfy certain laws. In Haskell terms:

\begin{lstlisting}
     fmap id == id
fmap (g . h) == (fmap g) . (fmap h)
\end{lstlisting}

\begin{notelist}
    \item Mapping the identify function over the contents of a \code{Functor} just gives back the original \code{Functor}.
    \item \code{fmap} distributes over function composition.
    \item Ultimately, these two laws just mean that a well-behaved \code{Functor} instance only operates on the
          contents of the \code{Functor}, leaving its structure unchanged.
\end{notelist}

Consider this badly-behaved instance definition for lists taken from \cite{typeclassopedia}.

\begin{lstlisting}
instance [] where
    fmap g []     = []
    fmap g (x:xs) = g x : g x : fmap g xs
\end{lstlisting}

\begin{notelist}
    \item This implementation of \code{fmap} duplicates all the output values: \code{fmap (+1) [1, 2, 3]} returns
          \code{[2, 2, 3, 3, 4, 4]}.
    \item The first law is broken because \code{fmap id [1, 2, 3]} returns \code{[1, 1, 2, 2, 3, 3]} rather than 
          \code{[1, 2, 3]}.
    \item The second law is broken because \code{fmap ((+1) . (*2)) [1,2,3]} returns \code{[3, 3, 5, 5, 7, 7]} rather than
          \code{[3, 5, 7]}.
\end{notelist}

Although Haskell's type system is quite powerful, it is in general undecideable whether a \code{Functor} instance satisfies
the two laws described above, so that requirement cannot be checked at compile time. Since other Haskell code in the standard
libraries and elsewhere will expect new \code{Functor} instances to be well-behaved, it is the responsibility of
the programmer to prove, at least to their own satisfaction, that their implementation satisfies those laws.  We will look at
more typeclasses later on, each with their own laws, and this caveat applies to them as well.

\section{Applicative Functors}

When we looked at the \code{Functor} type class, we saw in \code{fmap} a way for us to lift
normal functions into the domain of computational contexts, the \code{Functor} instances, where
they can operate on values in those contexts.

But recall that in Haskell, functions are themselves first-class values. So how do we use a function
that is itself in a computational context? This question is answered by the concept of applicative
functors, realized in Haskell with the \code{Applicative} type class.

Consider this scenario. We are given an \code{Integer} with the possibility that it might not
actually be there, i.e., \code{Maybe Integer}. We are also given a function to apply to that value,
again with the possibility that it might not actually be there, i.e., \code{Maybe (Integer -> Integer)}.
Of course, since either the function or the parameter might be \code{Nothing}, the return value needs
to be able to propagate this possibility of failure. How would we accomplish that?

\begin{lstlisting}
maybeApplyToMaybe :: Maybe (Integer -> Integer) -> Maybe Integer -> Maybe Integer
maybeApplyToMaybe Nothing _ = Nothing
maybeApplyToMaybe _ Nothing = Nothing
maybeApplyToMaybe (Just f) (Just x) = Just \$ f x
\end{lstlisting}

There are three cases:

\begin{notelist}
    \item Line 2: When we get no function to apply, the parameter does not matter: the result is \code{Nothing}.
    \item Line 3: When we get no parameter to apply the function to, the function does not matter: the result is \code{Nothing}.
    \item Line 4: When we actually get a function and an parameter, we can unwrap them from their \code{Maybe} wrapper,
          apply the function to the parameter, and return the result wrapped back up.
\end{notelist}

What happens if we want to use a function with two or more parameter in this way? We do not actually have to write any more
code: Haskell's default of curried functions gives us native partial application of functions and \code{maybeApplyToMaybe}
gets that for free.

Say we have a function \code{add :: Integer -> Integer -> Integer}. The expression \code{add 5} has type \code{Integer -> Integer}.

If we look back at line 4 of the previous example, we apply our unwrapped function to the unwrapped parameter, and return the
result wrapped back up. Since the partial application \code{add 5} returns a function of type \code{Integer -> Integer}, 
\code{maybeApplyToMaybe (Just add) (Just 5)} returns a value of type \code{Maybe (Integer -> Integer)}. 

\subsection{The \code{Applicative} Type Class}

We saw in the previous section that \code{map} for lists and \code{applyToMaybe} shared a common pattern, which led us to the
general \code{Functor} type class. The same idea works here, and the resulting type class is called \code{Applicative}, short for
\textbf{applicative functor}.

\begin{lstlisting}
class Functor f => Applicative f where
    pure  :: a -> f a
    (<*>) :: f (a -> b) -> f a -> f b
\end{lstlisting}

The \code{Applicative} type class has its own class constraint: every instance of \code{f} of \code{Applicative} must also
be an instance of \code{Functor}. In fact, most of the standard library's \code{Functor} instances are also \code{Applicative}s
as well.

Importantly, though, \code{Applicative} and its functions are not included in the Prelude and must be imported manually from
the \code{Control.Applicative} module.

\subsection{\code{Applicative}'s Functions}

If we look back at \code{maybeApplyToMaybe :: Maybe (Integer -> Integer) -> Maybe Integer -> Maybe Integer}, we see simply
a monomorphic instance of the more general, polymorphic type of \code{(<*>)}. In fact, Haskell would have inferred a more
general type for \code{maybeApplyToMaybe}: \code{Maybe (a -> b) -> Maybe a -> Maybe b} and our implementation is essentially 
the standard library's implementation of \code{(<*>)} for \code{Maybe}.

That means we can scrap our 17 character function name and rewrite \code{maybeApplyToMaybe (Just (+5)) (Just 2)} as 
\code{Just (+5) <*> Just 2}.

In general, \code{(<*>)} is the function (used infix like an operator) that takes a function in some \code{Applicative} context
\code{f} and a value in the same context and handles the plumbing of unwrapping the function and the value, applying the function
to the value, and returning the result wrapped back up in the \code{f} context.

We have not said much about the other half of the \code{Applicative} class, but it is quite simple and the type is very telling.
What makes sense for the type \code{a -> f a}?

We are getting a value of any type and returning a value of that type in the context described by \code{f}. Without knowing
anything about the type \code{a} of the parameter, we cannot modify it in any way. So, from the type alone, we can surmise that
\code{pure} probably injects a value into the context described by \code{f} in some default way.

What might \code{Maybe}'s implementation of \code{pure} look like? It is literally just \code{Just}!

\subsection{Building a Better \code{fmap}}

It might seem that all we really get from \code{Applicative} is a way to factor out the unwrapping required to apply a function
in a context to a value in a context, but there is more here.

Consider the operator \code{(<\$>) :: (a -> b) -> f a -> f b} provided by the \code{Control.Applicative} module.
It that takes a function, injects it into \code{f} with \code{pure}, and then uses \code{(<*>)} to apply it to a value
in \code{f}. 

The type of \code{(<\$>)} should look familiar: it is the same type as \code{fmap}. In fact, assuming both \code{Functor}
and \code{Applicative} laws (which we will see in a moment) the two are synonyms: \code{g <\$> pure x == fmap g \$ pure x}.

What we really get out of \code{Applicative} is a better version of \code{fmap}.

Suppose we called \code{fmap (+) (Just 3)}. This is not an error, we will just partially apply \code{(+)} to the wrapped
value \code{3} and get back basically \code{Just (3+)}, a function value inside a \code{Maybe} context, exactly where 
we were at the beginning of this section.

Now that we have seen how \code{Applicative}s work, we have the tools needed to finish \code{fmap}ping a function 
with two parameters: \code{fmap (+) (Just 3) <*> Just 5}, in \code{Applicative} terms: \code{pure (+) <*> Just 3 <*> Just 5},
or even more idiomatically: \code{(+) <\$> Just 3 <*> Just 5}.

From a practical perspective, we could write a function that attempted to build a \code{PlaneTicket} value based on
optional \code{Section} and \code{MealOption} values:

\begin{lstlisting}
createPlaneTicket
    :: Maybe Section
    -> Maybe MealOption
    -> Maybe PlaneTicket
createPlaneTicket section meal
    = PlaneTicket <\$> section <*> meal
\end{lstlisting}

\code{createPlaneTicket} uses the language of \code{Applicative} to succinctly lifts the \code{PlaneTicket} data constructor
into the \code{Maybe} context where the \code{MealOption}s and \code{Section}s might not exist.

In fact, the type that Haskell would actually infer for \code{createPlaneTicket} is
\code{(Applicative f) => f Section -> f MealOption -> f PlaneTicket} and would work for any \code{Applicative}, such as 
lists, which we will take a look at shortly.

\subsection{\code{Applicative} Laws}

There are four laws that \code{Applicative} instances should follow, with the same motivations and caveats described when
we discussed the \code{Functor} laws.

\begin{notelist}
    \item \textbf{Identity}: \code{pure id <*> v == v}
    \item \textbf{Homomorphism}: \code{pure g <*> pure x == pure (g x)}
    \item \textbf{Interchange}: \code{g <*> pure x == pure (\$ x) <*> g}
    \item \textbf{Composition}: \code{g <*> (h <*> k) == pure (.) <*> g <*> h <*> k}
    \item \textbf{Functor Instance}: \code{fmap g x == pure g <*> x}
\end{notelist}

The \textbf{identity law} can be thought of as putting an upper bound on what can actually happen inside the plumbing of
the \code{Applicative} implementation. If that plumbing does anything that fails to preserve identity, it
is not a proper \code{Applicative}.

The intuition behind the \textbf{homomorphism law} is that these operations are just lifting function application into
\code{Applicative} contexts. If we have a function \code{f} and a value \code{x}, inject each into the context
via \code{pure} and apply them via \code{(<*>)}, we should get the same thing as if we had injected
\code{f x} into the context directly.

The \textbf{interchange law} is a bit tricky. To start, remember that the \code{(\$)} operator is just function application
with a very low precedence. So \code{(\$ y)} is a function that takes a function and applies it to \code{y}.
What the interchange law is trying to express is that the order in which we evaluate the function and its 
parameter should not matter in a proper \code{Applicative} instance.

We can think of the \textbf{composition law} as formalizing an associative property for \code(<*>) in terms 
of Haskell's standard function composition operator \code{(.)}.

Finally, the \textbf{functor instance law} describes how an \code{Applicative} instance should behave relative to
its \code{Functor} instance and is required for the equivalence between \code{(<\$>)} and \code{fmap} to hold for
an \code{Applicative} instance.

\cite{typeclassopedia}[section 4.2] offers a bit more detail on the \code{Applicative} laws.

\subsection{\code{Applicative} and Lists}

When we looked at \code{Functor}s, our two canonical examples were \code{Maybe} and lists, but we have not really mentioned
lists yet in this section. The problem is not that lists are not \code{Applicative}s, but that there are two perfectly
reasonable ways to implement the \code{Applicative} instance for lists!

Lists are a context that support zero or more values. So suppose we had a list of functions and wanted to apply them
(in the \code{Applicative} sense) to some values also in a list context: 

\begin{lstlisting}
[(+1), (*2), (^3)] <*> [4, 5, 6]
\end{lstlisting}

We could certainly interpret this as pair-wise application, applying \code{(+1)} to \code{4}, \code{(*2)} to \code{5}, etc.

However, recall that we could view lists not just as a container of zero or more values but as a kind of non-deterministic value
where \code{[4, 5, 6]} represents a value that might be any one of those numbers. In this interpretation, it might make more
sense to do apply each function from the left-hand list to each value value in the right-hand list.

The result is a list containing the possible values when a non-deterministic function is applied to a non-deterministic
value, yielding a total of 9 possible values in this example.

In fact, the Haskell library's \code{Applicative} instance for lists uses the latter interpretation. The implementation looks like this:

\begin{lstlisting}
instance Applicative [] where
    pure x    = [x]
    gs <*> xs = [ g x | g <- gs, x <- xs ]
\end{lstlisting}

\begin{notelist}
    \item \code{pure} injects a value into the list context by creating a singleton list containing that value.
    \item \code{(<*>)} applies each function from the left-hand list to each value in the right-hand list as discussed. It does
          so via Haskell's \textbf{list comprehension} syntax.
\end{notelist}

What about the pair-wise version of \code{Applicative} for lists? Due to language constraints, the list type cannot have
two implementations for the same type class. Instead, Haskell offers type called \code{ZipList} that wraps a normal
list but offers a different \code{Applicative} instance:

\begin{lstlisting}
newtype ZipList a = ZipList { getZipList :: [a] }
 
instance Applicative ZipList where
   pure x = repeat x
   (ZipList gs) <*> (ZipList xs) = ZipList (zipWith (\$) gs xs)
\end{lstlisting}

\begin{notelist}
    \item The \code{newtype} keyword defines a type synonym that is checked at compile time but discarded so there is no
          run-time overhead. A \code{ZipList} can never be used as a normal list, but there is no additional overhead.
          Record syntax is used here to automatically create a function \code{getZipList} to translate normal lists to
          \code{ZipList}s.
    
    \item \code{zipWith} takes two lists and applies a function pair-wise to the elements of those lists. In this case
          the function is \code{(\$)}, which we have seen previously. So the \code{ZipList} instance of \code{Applicative}
          is implementing pair-wise application.

    \item Because \code{zipWith} truncates the result to the length of the shorter of its two parameter, it makes sense
          for \code{pure} to inject values into the \code{ZipList} context by creating an infinite list via \code{repeat}.

    \item If we used the same \code{pure} implementation as normal lists, \code{pure g <*> [1..] == [g 1]} and the functor
          instance law no longer holds.
\end{notelist}

\subsection{Summary}

In this section we have looked at applicative functors and Haskell's \code{Applicative} type class. We have seen how
\code{Applicative} offers an abstraction for applying functions even when the functions themselves were wrapped up in
a context just as \code{fmap} allowed us to apply bare functions to values in a context.

In the next section we will discuss the \code{Monad} type class and see how it further extends the notion of computational
contexts that we have built up via \code{Functor} and \code{Applicative}.


\section{Monads}

Consider the following Java-like pseudocode:

\begin{lstlisting}
String emailDomain = user.getContactInfo()
                         .getEmailAddress()
                         .getDomain();
\end{lstlisting}

We have several domain objects:

\begin{notelist}
    \item \code{User}, which has method \code{getContactInfo()} that returns the user's contact information with the type 
    \item \code{ContactInfo}, which as a method \code{getEmailAddress()} that returns the associated email address with type
    \item \code{EmailAddress}, which has a method \code{getDomain()} which returns the the domain portion of that email address as a \code{String}.
\end{notelist}

Now suppose that a \code{User}'s \code{ContactInfo} is optional so that \code{getContactInfo()} might return \code{null}. Likewise,
a \code{ContactInfo} record's \code{EmailAddress} is optional so that \code{getEmailAddress()} might return \code{null}.

That makes the above code snippet dangerous. The \code{getEmailAddress()} and \code{getDomain()} calls could be performed on null
references, causing a \code{NullPointerException}. If uncaught, the program crashes.

We could try this:

\begin{lstlisting}
String emailDomain;
ContactInfo contactInfo;
EmailAddress emailAddress;

contactInfo = user.getContactInfo();

if ( contactInfo != null )
{
    emailAddress = contactInfo.getEmailAddress();

    if ( emailAddress != null )
    {
        emailDomain = emailAddress.getDomain();
    }
    else
    {
        emailDomain = null;
    }
}
else
{
    emailDomain = null;
}
\end{lstlisting}

We have managed to propagate possible \code{null} values through the chain of method calls, but at the cost of a great deal of 
boilerplate code.

Of course, we have already seen how Haskell's \code{Maybe} type lets us encode the possibility of \code{null}-like values
explicitly. Suppose we have analagous Haskell types and functions. The above code translates to something like:

\begin{lstlisting}
userEmailDomain :: User -> Maybe String
userEmailDomain user = case getContactInfo user of
    Nothing          -> Nothing
    Just contactInfo -> case getEmailAddress contactInfo of
        Nothing    -> Nothing
        Just email -> getDomain email
\end{lstlisting}

Although we are now explicit about the possibility of a null result, we have not really addressed the issue of boilerplate.
If we had to chain together even more calls, our code would quickly stair-step right off the screen.

We can see a pattern, however. When we apply \code{getContactInfo} to \code{user}, we do a pattern match. If we got an
actual value (the \code{Just} case), we take that value and pass it on to the \code{getEmailAddress} call. In the \code{null}
case, though, we short-circuit the chain of function calls and just return \code{Nothing}. The same strategy is used when
we try to pass the result of \code{getEmailAddress} into \code{getDomain}.
 
We can generalize this pattern:

\begin{lstlisting}
chainMaybe :: Maybe a -> (a -> Maybe b) -> Maybe b
Nothing  `chainMaybe` f = Nothing
(Just x) `chainMaybe` f = fx
\end{lstlisting}

Note: Here we juse backticks around the function name to cause Haskell treat the function like an infix operator, not unlike
using parentheses around an infix operator causes Haskell to treat it like a regular, prefix function.

Having factored out the pattern matching to handle both cases, we can rewrite our stair-stepped chain of function calls:

\begin{lstlisting}
userEmailDomain :: User -> Maybe String
userEmailDomain user =
(Just user) `chainMaybe` getContactInfo
            `chainMaybe` getEmailAddress
            `chainMaybe` getDomain
\end{lstlisting}

Now our code is as readable as the original Java chain of method calls, but with the \code{null}-safety of the clunky second attempt.

\subsection{The \code{Monad} Type Class}

Not surprisingly, Haskell offers a type class to describe this same pattern in polymorphic terms:

\begin{lstlisting}
class Monad m where
    (>>=)  :: m a -> (a -> m b) -> m b
    (>>)   :: m a -> m b -> m b
    return :: a -> m a 
    fail   :: String -> m a

    m >> k = m >>= \_ -> k
\end{lstlisting}

\begin{notelist}
    \item \code{(>>=)} is the monadic chaining operator. It is just a polymorphic, infix equivalent of \code{chainMaybe}, which is
          how \code{Maybe}'s implementation of \code{(>>=)} is defined.
    \item \code{(>>)} can be thought of as a chaining operator where the value passed into the right-hand function is simply ignored.
          It is given a default implementation in terms of \code{(>>=)} on line 5.
    \item \code{return} is the \code{Monad} class's general method for injecting a value into a monadic wrapper. Code that uses
          the \code{Monad} interface cannot use specific constructors like \code{Just} on line 3 of our final \code{userEmailDomain}
          implementation, so \code{Monad} instances implement \code{return} to define the behavior. If \code{return} sounds familiar,
          it is because it is actually identical to \code{pure} from \code{Applicative}. We will discuss this further in a moment.
    \item \code{fail} offers a way for \code{Monad} instances to short-circuit evaluation when a computation has failed. Use of 
          \code{fail} is discouraged in general because some \code{Monad}s, including \code{IO}, implement \code{fail} by raising a fatal error.
          \code{Maybe}'s implementation of \code{fail} returns \code{Nothing}.
\end{notelist}

\subsection{\code{Monad} and \code{Applicative}}

We mentioned before that \code{return} and \code{pure} were basically identical. In fact, conceptually,
\code{Applicative} is a superclass of \code{Monad}. In fact, the hierarchy from \code{Functor} to \code{Applicative}
to \code{Monad} represents progressively more flexible operations on with values inside some context.

However, while \code{Functor} and \code{Monad} were part of the Haskell standard library as described in the Haskell 98 standard \cite{haskell98},
applicative functors were not introduced until 2008 in the paper \emph{Applicative Programming with Effects} \cite{applicative} by McBride and
Paterson.

Altering the standard library's definition of \code{Monad} to include an \code{Applicative} class constraint would break existing user-defined
\code{Monad} instances that did not offer an \code{Applicative} implementation. However, a proposal to make this change is likely to be implemented
in the near future. For that reason, newer versions of GHC will issue warnings when \code{Monad} instances are declared without accompanying
\code{Applicative} instances. 

In the mean time, all the \code{Monad} instances we will discuss have accompanying \code{Applicative}.

\subsection{The List Monad}

Now that we have seen the basic definition of the \code{Monad} type class and seen a simple instance in \code{Maybe}, we
can look at a more complex example: lists.

Recall that lists can be viewed as simple containers or as computational contexts supporting non-deterministic values.
The list monad is based on this non-deterministic value perspective. 

\begin{notelist}
    \item The \code{Functor} instance for lists lifted function application into the domain of non-deterministic values.
    \item The \code{Applicative} instance for lists (i.e., not the \code{ZipList} instance) introduces the ability to
          apply non-deterministic function values to non-deterministic values.
    \item Finally, we can think of the \code{Monad} instance for lists as lifting computation in general into the
          domain of non-deterministic values.
\end{notelist}

Here is how the list instance of \code{Monad} is defined:

\begin{lstlisting}
instance Monad [] where  
    return x = [x]  
    xs >>= f = concat (map f xs)  
    fail _   = []  
\end{lstlisting}

\begin{notelist}
    \item \code{return} is, again, equivalent to \code{pure} and simply returns a singleton list containing the given element.
    \item \code{fail} returns the empty list.
    \item \code{(>>=)} first maps \code{f :: a -> [a]} over \code{xs} yielding a value of type \code{[[a]]}, i.e., a list of lists.
          Then \code{concat :: [[a]] -> [a]} concatenates each of those lists into a single list. For example,
          \code{concat [[1, 2], [3, 4]] == [1, 2, 3, 4]}.
\end{notelist}

Suppose we wanted to work with a square root function that non-deterministically returned both the postive and negative
square roots of its parameter:

\begin{lstlisting}
sqrt' :: Double -> [Double]
sqrt' x = [sqrt x, negate $ sqrt x]
\end{lstlisting}

Now, when we evaluate \code{[4.0, 9.0] >>= sqrt'}, we map \code{sqrt'} over the list, yielding \code{[[2.0, -2.0], [3.0, -3.0]]}. 
Then \code{concat} is applied, yielding \code{[2.0, -2.0, 3.0, -3.0]}.

Suppose we try to evaluate \code{[16.0, 81.0] >>= sqrt' >>= sqrt'} (note that \code{(>>=)} is left associative) so we can simplify
this expression:

\begin{lstlisting}
([16.0, 81.0] >>= sqrt') >>= sqrt' 
== [4.0, -4.0, 9.0, -9.0] >>= sqrt'
== [2.0, -2.0, NaN, NaN, 3.0, -3.0, NaN, NaN]
\end{lstlisting}

When \code{sqrt'} is applied to one of the negative intermediate values, the result is \code{NaN}, since we cannot take the
real square root of a negative. We would like to extend our non-deterministic square root function to deal with that case.

\begin{lstlisting}
sqrt'' :: Double -> [Double]
sqrt'' x | x >= 0.0  = [sqrt x, negate $ sqrt x]
         | otherwise = []
\end{lstlisting}
 
We use \textit{guard clauses} to deal with the two cases. If the parameter is non-negative, yield two possible values, otherwise,
yield no values at all. 

Now, if we evaluate \code{[16.0, 81.0] >>= sqrt'' >>= sqrt''} we get \code{[2.0, -2.0, 3.0, -3.0]}. From the perspective of non-deterministic
values, these are the values that could result from applying \code{sqrt''} twice to the non-deterministic value \code{[16.0, 81.0]}.
When applying \code{sqrt''} to negative values, the return value of \code{[]} represents a path of computation that has failed, and no
trace of it shows up in the final result.

\subsection{The \code{Monad} Laws}

Like \code{Functor} and \code{Applicative}, there are laws that govern how \code{Monad} instances should behave:

\begin{notelist}
    \item \textbf{Left identity}: \code{return a >>= f == f a}
    \item \textbf{Right identity}: \code{m >>= return == m }
    \item \textbf{Associativity}: \code{(m >>= f) >>= g == m >>= (\\x -> f x >>= g x)}
\end{notelist}

The left and right identity laws describe are primarily concerned with the neutral behavior of \code{return}. The associativity
law, with the behavior of sequences of monadic operations linked together with \code{(>>=)}.

Interestingly, none of these laws look much like identity or associativity as we might recall from algebra. However, if we introduce
a new (but related) operator, they do:

\begin{lstlisting}
(>=>) :: Monad m => (a -> m b) -> (b -> m c) -> a -> m c
(f >=> g) x = f x >>= (\y -> g y)
\end{lstlisting}

The operator \code{(>=>)} acts like the standard function composition operator \code{(.) :: (a -> b) -> (b -> c) -> a -> c}
and if we rewrite the above laws in terms of \code{(>=>)}, the names look far more appropriate:

\begin{notelist}
    \item \textbf{Left identity}: \code{return >=> f == f}
    \item \textbf{Right identity}: \code{f >=> return == f }
    \item \textbf{Associativity}: \code{(f >=> g) >=> h == f >=> (g >=> h)}
\end{notelist}

\subsection{The \code{State} Monad}

In the imperative programming paradigm, our programs essentially operate by mutating global state. To swap the values of two variables 
\code{a} and \code{b}, we might store the value in \code{a} into a temporary variable \code{t}, store the value in \code{b} into \code{a},
then store the value in \code{t} into \code{b}. Our programs just shuffle bit patterns around in memory.

In the end, Haskell programs are doing the same thing, but we are interested in expressing our programs in terms of higher-level 
constructs. However, the mutating state is a powerful tool and the \code{State} monad allows us to do emulate this style of
programming by providing the framework through which a value of some type can be passed through a sequence of monadic operations,
possibly being replaced along the way. We do not actually write values to locations in memory, but the end result is similar, while
being built on the same monadic abstraction we have looked at so far.

Here is the definition of the \code{State} type and its \code{Monad} instance.

\begin{lstlisting}
newtype State s a = State { runState :: s -> (a,s) }  

instance Monad (State s) where  
    return x = State $ \s -> (x,s)  
    (State h) >>= f = State $ \s -> let (a, newState) = h s  
                                        (State g) = f a  
                                    in  g newState 
\end{lstlisting}

The \code{State} type constructor takes two type parameters: \code{s} is the type of the state value and \code{a} is
the result type. Just as we might call \code{Maybe Integer} an \code{Integer} value in \code{Maybe}'s context of possible
failure, \code{State String Integer} might be referred to as an \code{Integer} value in the context of a stateful
computation, where the state is a \code{String} value.

The \code{newtype} definition of \code{State} uses Haskell's \textbf{record syntax} to created a \textbf{named field}.
A \code{State} value is actually a wrapper around a function of type \code{s -> (a, s)}. The record syntax automatically
creates a function \code{runState} which pulls that function out of the wrapper so that it can be applied to an initial
state value of type \code{s}, returning the result of type \code{a} and the final state, again of type \code{s}.

The \code{State} type is polymorphic in two type variables, but recall that the definition of the \code{Monad} type class
had only a single type variable. As line 3 suggests, it might be more appropriate to think of it as the \code{State s} monad,
with each concrete type (\code{State [Integer]}, \code{State String}, etc.) as being separate, incompatible monads that
happen to have identical implementations.

\code{State}'s implementation of \code{return} takes a value \code{x} and returns a function (wrapped with the \code{State} constructor)
that, given any state value \code{s} returns \code{(x, s)}. We can think of the result of \code{return} in the \code{State} monad as a
stateful computation that simply returns a value and leaves the state untouched. This is compatible with our understanding of \code{return}
as a neutral operation that injects a value into as bare a context as possible. 

The implementation of \code{(>>=)} is a bit more involved:

\begin{notelist}
    \item [Describe the types of the two operands, State h and f]
    \item [Describe let clause, bindings for intermediate values]
    \item [Describe the result]
\end{notelist}

[Describe an example, using [Integer] as the state type, with push and pop operations]

\subsection{\code{do}-Notation}

\begin{notelist}
    \item Describe do-notation syntax, syntactic sugar, Left-arrow binding
    \item Rewrite State example using do-notation
    \item Desugaring do-notation
\end{notelist}

\subsection{The \code{IO} Monad}

\begin{notelist}
    \item Describe IO monad as mechanism for definining impure computations in a pure language
    \item IO values aren't "flagged" values, but lazy descriptions of computations that go out and get values from
          the outside world when finally executed.
    \item Examples
    \item Describe how userEmailDomain behaves if written as generic monadic operation and applied in list monad, IO monad
\end{notelist}


\input{parser-combinators.tex}
\section{Monad Transformers}

We have seen how monads describe a general framework for programming with side effects
in an otherwise pure functional language.

However, although we can use \code{IO} for input and output and \code{State} to track mutable state,
we cannot use them together to write code that does both at the same time. We would like to be able
to compose two monads so that we can write code that takes advantage disparate types of side effects.

We will look at one solution to this problem: \textit{monad transformers}. 

Consider the following problem. We want to read lines of text from the console until the user enters
a blank line while keeping a running count the number of lines read and saving the longest line entered so far.

We clearly need \code{IO} to read from the console, but updating the statistics we need seems like
a good use case for \code{State}. Since \code{State} is not strictly necessary, we can implement
this only using only \code{IO}:

\lstinputlisting{code/io-state.hs}

\begin{notelist}
    \item Line 1 introduces a record type to track the statistics we are interested in.
    \item Line 5 defines a function that will begin the process, starting with an initial state.
    \item On line 8 we define a recursive \code{IO} action called \code{runStats}. It takes a \code{Stats}
          value and yields a possibly updated \code{Stats} value.
        \begin{notelist}
            \item We get a line of text from the console, binding the result to the identifier \code{line}.
            \item If the line is empty, we stop and yield the statistics gathered so far. 
            \item In this context, we appear to be using \code{return} like we would in an imperative
                  language. However, in Haskell, \code{return} is just a function that wraps a 
                  value inside a monadic context and has nothing to do with terminating execution
                  of a procedure. We only use \code{return} because we need the result of this
                  expression to be of type \code{IO Stats}.
            \item If the line is not empty, we recursively call \code{runStats} with updated statistics.
        \end{notelist}
    \item Line 15 defines a helper function that takes a \code{String} and  updates a \code{Stats} value 
          accordingly. It increments the count and keeps the given \code{String} if it is longer. Note
          that we are not mutating the original \code{Stats} value, only using its constituents to build
          a new one.
\end{notelist}

This solution works, but it has the drawback that we were responsible for keeping track of the state value,
explicitly passing it through recursive calls to \code{runStats}. We want compose \code{IO} and \code{State}
to give us a monad that gives us mutable state while still allowing the I/O operations we require.

\subsection{The Monad Transformer Library}

The Haskell standard library includes a framework for composing two or more monads, called the Monad Transformer
Library. 

Monad transformers work by layering an interface for monadic operations of one type on top of a base monad.
We saw earlier how to use the \code{State} monad to write stateful code that manipulated a stack of integers.
Here we will extend that example, wrapping the \code{IO} monad with \code{State} to give us state manipulation
and I/O at the same time.

\lstinputlisting[lastline=6]{code/monad-transformers.hs}

\begin{notelist}
    \item We need to import \code{Control.Monad.Trans} to access some basic transformer-related functions, namely \code{lift}
          which we will see below.
    \item The \code{Control.Monad.Trans.State} module gives us the transformer for \code{State}.
    \item On line 4, we define a type synonym for the state values our combined monad will track.
    \item On line 6, we define a type synonym, \code{StatePlusIO}, for our combined monad. 
    \item We build our combined monad using the type \code{StateT s m a}:
        \begin{notelist}
            \item By convention the transformer equivalent of a monad is suffixed with ``\code{T}''. Thus, \code{State}'s
                  transformer equivalent is \code{StateT}.
            \item The first type variable, \code{s}, is the type for the state values, \code{Stack} in the example.
            \item The second type variable, \code{m}, is the type of the underlying monad, \code{IO} in the example.
            \item The third and final type variable, \code{a}, is just the result type of values yielded by operations
                  in the monad. We do not specify it here, since we may want operations of type \code{StatePlusIO Integer}
                  or \code{StatePlusIO ()}, etc.
        \end{notelist}
\end{notelist}

Now we have a new monadic type that has the same stack manipulation categories we saw earlier but with the capability of
performing I/O. We can write some basic actions to interact with this new monad:

\lstinputlisting[firstline=8,lastline=37]{code/monad-transformers.hs}

\begin{notelist}
    \item The functions \code{push} and \code{pop} operate just as they did before, but use \code{lift} to
          turn the \code{IO} action \code{putStrLn} into an action in the \code{StatePlusIO} monad.
    \item The general type of \code{lift} is \code{m a -> t m a}. Haskell can infer both the inner and
          outer monad based on the environment that \code{lift} is used in.
    \item In this case, \code{m} is \code{IO} and \code{t} is \code{StateT Stack}.
    \item So \code{lift} takes \code{putStrLn :: String -> IO ()} and gives us \code{String -> StateT Stack IO ()}.
    \item The \code{readPush} action uses our stateful framework plus I/O to read an integer from the console and
          push the result onto the stack. Again, \code{lift} brings the \code{readLn} action from the underlying
          \code{IO} monad up into our combined monad.
\end{notelist}

Just as with the examples we saw using a simple \code{State} monad, the actions we define for \code{StatePlusIO}
will not do anything until we run them with \code{runStateT}:

\begin{lstlisting}
ghci> runStateT calculator []
? 3
Pushed 3
? 5
Pushed 5
Popped 5
Popped 3
Pushed 8
((),[8])
\end{lstlisting}

To run a \code{State} action, we used \code{runState :: State s a -> s -> (a, s)}, which took a \code{State}
action and an initial state and returned the result along with the final state. The type of \code{runStateT} has a
slightly different type: \code{StateT s m a -> s -> m (a, s)}. We still provide a monadic action (in this case,
built from the \code{StateT} transformer) and an initial state, but the resulting pair is wrapped inside the
underlying monad.

In the case of \code{StatePlusIO}, the result of \code{runStateT} is a value in the \code{IO} monad.

The monad transformer library allows us to compose more than just two monads. In the following example,
we will extend \code{StatePlusIO} with globally accessible, read-only configuration using \code{ReaderT}
to give our stack manipulation code a list of values that it will refuse to store:

\lstinputlisting[lastline=17]{code/more-monad-transformers.hs}

\begin{notelist}
    \item We add an additional import this time, \code{Control.Monad.Trans.Reader} for the \code{Reader}
          monad's transformer equivalent \code{ReaderT}.
    \item We also define a type synonym for our configuration value, in this case a list of integers.
    \item The definition of our new combined type is similar to before, but instead of layering
          \code{StateT} over just \code{IO}, we are building on top of the monad we get from 
          layering \code{ReaderT} over \code{IO}. We are composing three different monads
          to create the combined monad \code{StateReaderIO}.
    \item The \code{push} function receives the biggest change:
        \begin{notelist}
            \item In the \code{Reader} monad, the function \code{ask} yields the read-only value
                  carried along with the computation.
            \item Since \code{ReaderT} is not the top-level monad, we have to use \code{lift} 
                  to bring \code{ask} (an action in the \code{ReaderT Config IO} monad) up into
                  the full \code{StateReaderIO} monad.
            \item We bind this result to the identifier \code{blacklist} and if the value we are
                  attempting to push is in that list, we alert the user that the operation is
                  forbidden. Otherwise, we push the value onto the stack as before.
            \item We use a new function \code{liftIO} to lift the \code{IO} action into our
                  full combined monad. Because \code{lift} can only bring an action up
                  a single level in the monad stack, we would actually need \code{lift . lift}
                  (\code{lift} composed with itself) to bring an \code{IO} action up
                  two levels into \code{StateReaderIO}.
            \item Because \code{IO} is often the base monad on top of which several monad
                  transformers are layered, \code{liftIO} was included in the monad transformer
                  library to lift \code{IO} actions to the top level no matter where in the 
                  stack \code{IO} is actually located.
        \end{notelist}
\end{notelist}

Of course, we need a function to evaluate an action in our combined \code{StateReaderIO} monad.
Recall that \code{runStateT} ran our \code{StateT}-based action and resulted in a value within
the underlying \code{IO} monad.

In this case, \code{runStateT} will take a \code{StateReaderIO} action and an initial state, but
give us a value in the \code{ReaderT Config IO} monad. Thus we need to use \code{runReaderT} to
fully evaluate a \code{StateReaderIO} action to get a result in the \code{IO} monad.

Below we demonstrate \code{readPush} in our new monad, where \code{[1,2,3]} is the blacklist of values
that \code{push} will reject and \code{[]} is the initial state.

\begin{lstlisting}
ghci> runReaderT (runStateT readPush []) [1,2,3]
? 4
Pushed 4
((),[4])

ghci> runReaderT (runStateT readPush []) [1,2,3]
? 2
Forbidden: 2
((),[])
\end{lstlisting}


\subsection{Monadic Parsing}

Parsing text is a frequently encountered problem in computing. Simple recursive descent parsing is almost
identical to computations in the \code{State} monad. A parser for some type \code{t} takes some initial 
state, the input buffer, and returns a value of type \code{t} and the unconsumed input.

However, parsing does not always succeed. For this reason, we would like to extend the \code{State} monad
with the possibility of failure. In this case, we would like failures to come with some explanation. Rather
than use \code{Maybe}, we will use \code{Either e t}, where \code{e} is the type of the error value and
\code{t} is the type of a successful result.

\code{Either} has kind \code{* -> * -> *}, so it cannot be a monad. However, if we apply one of the type
parameters, we get a proper monad: \code{Either e} for some error type \code{e}.

\lstinputlisting{code/either.hs}

\begin{notelist}
    \item \code{Either} has two data constructors, \code{Left} for errors, \code{Right} for values.
    \item The \code{Functor}, \code{Applicative}, and \code{Monad} instances are all analogous to the ones we
          saw for \code{Maybe}, but rather than dealing with \code{Nothing} values, the failure case is  \code{Left}
          wrapping an error value.
    \item We avoid \code{Monad}'s \code{fail} function since it takes a \code{String} and we want to be able
          to use any type for errors. The default implementation (causing a fatal error) will be used. As a result,
          our code will use \code{Right} to signal errors.
\end{notelist}

Just as the \code{State} monad has its transformer equivalent \code{StateT}, \code{Either} has its transformer,
called \code{ExceptT}. We will use \code{ExceptT} on top of \code{State} to build our parsing monad.

\lstinputlisting[lastline=12]{code/parsing.hs}

\begin{notelist}
    \item On line 7, we define \code{ParseError} as a synonym for \code{String}.
    \item On line 9, We define a type synonym for our parser type. \code{Parser a} is much easier to read
          than the alternative, but ultimately our parser type is just stateful computation modified with failure 
          with error messages.
    \item Lines 11-12 define the function to drives our \code{Parser} computations. It takes a parsing computation
          and an initial state and returns the result (a value or error message) plus the unconsumed input. 
\end{notelist}

Now we can write some basic parsing primitives:

\lstinputlisting[firstline=14,lastline=46]{code/parsing.hs}

\begin{notelist}
    \item \code{runParser} is a convenience function that composes running a \code{State} computation with 
          the \code{ExceptT} monad transformer on top of it. 
    \item The function \code{char} tries to parse a single character \code{c}. We \code{get} the current remaining
          parse buffer. If the buffer is empty, our parse fails. If there is at least one character in the buffer, we check 
          whether it is equal to \code{c}. If so, update the state with the remainder of the buffer and yield the character.
          Otherwise, our parse fails.
    \item The \code{string} parser tries to parse a given string. We implement this using the general monadic function
          \code{mapM :: Monad m => (a -> m b) -> [a] -> m [b]}. This function sequences the application of a function
          over each element in a list, yielding the collecting the results. In this case, the function is 
          \code{char :: Char -> Parser Char}. Since \code{String} is equivalent to \code{[Char]}, we sequence
          the parsing of each character in the string, yielding the parsed string. If parsing any of the individual
          characters fails, parsing the entire string fails.
    \item The choice operator \code{(<|>) :: Parser a -> Parser a -> Parser a} attempts the left-hand parser using \code{catchE} which
          runs a computation that might raise an exception and, if an exception occurs, passes the exception value into another action.
          In this case, we ignore the exception value, replace the initial parser state and try the right-hand parser.
    \item \code{many} parses zero or more repetitions of the given parser, yielding a list of the results. It use \code{(<|>)} chooise
          between no parses (\code{return []} on the right) or one or more. This definition uses applicative notation to
          lift \code{(:)} to build lists of parse results in a way that is more succinct than the equivalent \code{do}-notation.
    \item \code{manyOne} is like \code{many}, but requires at least one successful application of the given parser.
\end{notelist}

With these simple primitives and the general library functions we get for \code{Applicative} and \code{Functor} we can
write recursive descent parsers. The following simple parser parses nested parentheses, yielding a tree structure:

\lstinputlisting[firstline=48]{code/parsing.hs}

\begin{lstlisting}
ghci> runParser parens "()"
(Right Leaf,"")

ghci> runParser parens "(()(()))"
(Right (Node [Leaf,Node [Leaf]]),"")

ghci> runParser parens ""
(Left "Unexpected end of input","")
\end{lstlisting}


\begin{thebibliography}{99}
\bibitem{haskell98}
  Simon Peyton Jones, et al.,
  \href{http://www.haskell.org/onlinereport/index.html}{\emph{Haskell 98 Language and Libraries: The Revised Report}},
  2002.
\bibitem{haskellhistory}
  Paul Hudak, et al.,
  \href{http://www.scs.stanford.edu/\~dbg/readings/haskell-history.pdf}{``A History of Haskell: Being Lazy With Class''},
  2007.
\bibitem{hope}
  R.M. Burstall, D.B. MacQueen, D.T. Sannella,
  \href{http://homepages.inf.ed.ac.uk/dts/pub/hope.pdf}{``Hope: An Experimental Applicative Language''},
  1980.
\bibitem{derivabletype}
  Ralf Hinze, Simon Peyton Jones,
  ``Derivable Type Classes'',
  \emph{Proceedings of the Fourth Haskell Workshop}, 227--236,
  2000
\bibitem{typeclassopedia}
  Brent Yorgey,
  ``The Typeclassopedia'',
  \href{http://www.haskell.org/wikiupload/8/85/TMR-Issue13.pdf}{\emph{The Monad.Reader}}, 13, 17-68,
  2009
\end{thebibliography}

\end{document}
