\documentclass[12pt]{article}
% \usepackage{palatino}
\usepackage{epsfig}
\usepackage{epstopdf} % for \psfig{.eps}
\usepackage{amssymb} % for \varnothing
\usepackage{amsmath} % for \pmod
\usepackage{url} % to split long names
\usepackage{color}
\usepackage{listings}
\setlength{\topmargin}{-0.1in}
\setlength{\textheight}{8.0in}
% \newcommand\note[1]{{\color{red} #1}}
\pagestyle{myheadings}
\markboth{Haskell Lecture Notes}{Haskell Lecture Notes}
\newcommand\note[1]{\raggedright\fbox{#1}}
% \newcommand\ind{\hspace*{.3in}}
\newcommand\codebold[1]{\texttt{\textbf{#1}}}
\def\>{\hspace*{0.2in}}
\newenvironment{notelist}{\begin{list}
   {$\bullet$}
   {\setlength{\itemsep}{0in}}}
   {\end{list}}
    
% Options for code listings
\lstset{
    frame=single,
    language=Haskell,
    numbers=left
}

\author{Zachary May}
\title{Haskell Lecture Notes}
\begin{document}
\maketitle

\section{Introducing Haskell}
% \note{Lecture 1, 1/16/2014}
\begin{notelist}
\item Haskell is a \textbf{statically-typed}, \textbf{non-strict}, \textbf{pure} \textbf{functional}
      programming language.
	\begin{notelist}
	\item \textbf{Functional:} Conceptually, computation via the application of functions to arguments rather than sequential instructions
          manipulating values in memory. Functions are first-class values.
	\item \textbf{Pure:} The functions in question are more like mathematical functions than ``procedures''. They map values in an input domain
          to values in an output domain. Pure functions have no ``side effects''. This is a big win for reasoning about and testing our code.
	\item \textbf{Non-strict:} By default, Haskell uses a ``lazy'' evaluation strategy. Expressions do not need to be evaluated until the 
          results are needed. For example, Haskell can cleanly represent infinite lists because the language never tries to fully evaluate it.
    \item \textbf{Statically-typed:} The type of very expression is known at compile time, preventing run-time errors caused by type
          incompatibilities. This prevents things like Java's \codebold{NullPointerException}, because a function that claims it returns
          a value of a specific type has to live up to that promise and returning the Haskell equivalent of \codebold{null} is a compile-time
          type error. Additionally, Haskell makes use of a technique called \textit{type inference} to figure out the types of most things
          without needing explicit type annotations.
	\end{notelist}
\end{notelist}
\pagebreak

\section{A Sample Program}

\begin{lstlisting}
sumSquares :: Integer -> [Integer] -> Integer
sumSquares count numbers =
    sum (take count (map (^2) numbers))

printSquares :: IO ()
printSquares =
   print $ sumSquares 10 [1..]
\end{lstlisting}

\begin{notelist}
    \item The type signature in line 1 describes a function with two arguments: an \codebold{Integer} and a list
          of \codebold{Integer}s. It returns an \codebold{Integer}. In general, the type after the last \codebold{->}
          is the return value, and the others are the arguments.
    \item Haskell's type inference would actually figure out a more general type than what we provided in the type
          annotation on line 1. Haskell programmers usually include type annotations as a form of machine-checked
          documentation.
    \item Haskell functions are defined with an equational syntax as seen in lines 2-3: \codebold{sumSquares}
          applied to the arguments \codebold{count} and \codebold{numbers} is equal to the expression on the
          right-hand side of the equation.
    \item Line 3 shows us several examples of \textit{function application}. The parentheses here are for grouping
          only. The notation for function application is very lightweight.
    \item Working from the inside out, \codebold{map} applies a function to each element in a list, producing a new 
          new list containing the resulting values.
    \item \codebold{(\string^2)} is called a \textit{section}, a shorthand for an anonymous function whose argument
          ``fills in the blank'' for a binary operator, exponentiation in this case. So \codebold{map (\string^2)}
          transforms a list of numbers into their squares.
    \item Given a number $n$ and a list, \codebold{take} returns the first $n$ items of the list, or the whole list
          if it has fewer than $n$ elements. \codebold{sum} takes a list of numbers and returns the sum.
    \item Lines 5-7 define another function, \codebold{printSquares}. It takes no arguments, and its return type, \codebold{IO ()},
          is quite interesting.
    \item Look back at the type \codebold{[Integer]}. The list type \codebold{[]} is a
          \textit{parameterized} type. That is, ``list'' itself is not a concrete type, but a \textit{type constructor}.
          We need another type, the parameter, to create a concrete type. Similarly, \codebold{IO} is a type constructor.
    \item Here we instantiate it with the type \codebold{()}, the \textit{unit type} with only a single instance, also
          written as \codebold{()}. The type \codebold{IO ()} represents an \textit{I/O action} with no ``return''
          value. The \codebold{IO} type constructor is the technique Haskell uses to perform I/O, inherently impure,
          in a world of pure functions, via a far more general abstraction called the \textit{monad}.
    \item Consider the expression \codebold{[1..]}. This expression represents the infinite list of positive
          integers. Because Haskell is lazy, though, merely describing the value is not enough to force the run-time try
          evaluate the entire thing. In fact, in this case, evaluation of \codebold{[1..]} is only forced by
          \codebold{print} asking for the value produced by \codebold{sum}, which demands the values yielded by
          \codebold{take count}, and so on, on demand.
    \item Notice the \codebold{\$} in line 7. Pronounced ``applied to'', it is an operator with extremely low
          precedence that breaks up the very high precedence of function application, allowing us to avoid nesting
          parentheses. We could have defined \codebold{sumSquares} using \codebold{\$} as:
          \codebold{sum \$ take count \$ map (\string^2) numbers}. 
\end{notelist}

\end{document}
