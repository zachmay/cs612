\section{Maybe, Lists, and The Functor Typeclass}

Now let us consider a type defined in the Haskell Prelude, \code{Maybe}:

\begin{lstlisting}
data Maybe a = Nothing
             | Just a
\end{lstlisting}

\begin{notelist}
    \item \code{Maybe} is the Haskell version of the \textbf{option type}. It offers us a way to
          represent a value that might not exist. \code{Nothing} is the null value, and
          the \code{Just} constructor wraps an actual value.
    \item For example, we might want a function that parses an integer value from a string to
          have the return type \code{Maybe Integer}, since the parse might fail.
    \item Compare this to Java's type system where \code{null} is a possible value for any
          reference type. \code{null} is a \code{Person} even though \code{null} does not respond to  
          any of \code{Person}'s methods--or any methods at all!
    \item In Haskell, on the other hand, a function that claims to return a \code{Person} always returns
          a full-fledged \code{Person} (barring exceptional failure) and a function that sometimes
          returns a \code{null}-like value must declare that in its type, e.g., \code{String -> Maybe Integer}.
\end{notelist}

The safety we get when we use an option type is nice, but it comes with some inconvenience: If I 
have a value of type \code{Maybe Integer}, how do I add five to it? In general, how do I unwrap
a value of the form \code{Just x} to get at \code{x}? It is not difficult in principle:

\begin{lstlisting}
parseInteger :: String -> Maybe Integer
# Implemented elsewhere

example1 :: String -> Integer
example1 str = case parseString str of
                  Just x  -> x + 5
                  Nothing -> 0

example2 :: String -> Maybe Integer
example2 str = case parseString str of
                  Just x  -> Just (x + 5)
                  Nothing -> Nothing
\end{lstlisting}

\begin{notelist}
    \item However, this code has some shortcomings:
    \begin{notelist}
        \item In \code{example1}, we are making an assumption about how to handle the error case
              (returning zero if the parse failed) that is now interwoven with the independent process
              of adding five.
        \item Both examples repeat the code to test both the \code{Just} and \code{Nothing} case. Moreover,
              if we used \code{Maybe} frequently (which is encouraged), this would start to get rather
              annoying.
    \end{notelist}
\end{notelist}

We will reject \code{example1} because we really would like to maintain the orthogonality of
dealing with \code{Maybe} values and our actual operation. However we can use higher-order
functions to factor out repetition we see in analyzing \code{Maybe}s.

\begin{lstlisting}
applyToMaybe :: (a -> b) -> (Maybe a) -> (Maybe b)
applyToMaybe f Nothing  = Nothing
applyToMaybe f (Just x) = Just (f x)
\end{lstlisting}

\begin{notelist}
    \item \code{applyToMaybe} factors out handling the \code{Nothing} and \code{Just} cases of \code{Maybe}
          values.
    \item We also get some vocabulary for lifting normal function application into the world of \code{Maybe}
          values.
    \item In our add five example, we can now just use \code{applyToMaybe (+5) \$ parseString str}
\end{notelist}

Here is another example. We have seen Haskell's basic, homogeneous list type. It offers us a way to
represent a collection of zero or more values of some type.

We have also seen the function \code{map :: (a -> b) -> [a] -> [b]} that applies a function
to each element in a list and returns the results collected into a new list. 

\begin{lstlisting}
map :: (a -> b) -> [a] -> [b]
map f []     = []
map f (x:xs) = f x : (map f xs)
\end{lstlisting}

If we compare \code{applyToMaybe} and \code{map}, we see some important similarities:

\begin{notelist}
    \item Both functions have an empty case and a case where one or more values are unwrapped,
          a function applied, and the result(s) wrapped back up. 
    \item If we ignore the special case of Haskell's list type syntax, the functions have analogous types
          of the form \code{(a -> b) -> f a -> f b}
\end{notelist}

\subsection{The Functor Typeclass}

In fact, this pattern is codified in Haskell with the \code{Functor} type class:

\begin{lstlisting}
class Functor f where
    fmap :: (a -> b) -> f a -> f b
\end{lstlisting}

\begin{notelist}
    \item A \code{Functor} instance is always a polymorphic data type with a single type parameter.
    \item At one level we can think of \code{Functor}s as simply mappable containers.
    \item At another level, we can think of them as values in some sort of context, where \code{fmap}
          lifts function application into that new context.
\end{notelist}

\subsubsection{Functor Instances}

Here are some instances of the \code{Functor} type class:

\begin{notelist}
    \item \textbf{\code{Maybe}}
    \begin{notelist}
        \item We can think of \code{Maybe a} as a context representing a value of type \code{a} with the 
              possibility of failure.
        \item In this context, we can think of \code{fmap} as creating new functions that know how to
              propagate these failure states.
    \end{notelist}

    \item \textbf{Lists, i.e., \code{[]}}
    \begin{notelist}
        \item The implementation of \code{fmap} for lists is literally just the Prelude's \code{map} function.
        \item From the context perspective, we can think of lists as non-deterministic values; i.e., the
              result of applying the function \code{(* 2)} to the non-deterministic value that might be
              one of \code{[1, 2, 3]} would be the non-deterministic value that might be one of
              \code{[2, 4, 6]}.
    \end{notelist}

    \item \textbf{\code{Tree}}
    \begin{notelist}
        \item Mapping over a collection makes sense for trees, but what might \code{Tree} represent from the perspective
              of values in a context?
        \item Interestingly, while mapping over elements in a set seems reasonable enough, Haskell's \code{Set} type
              cannot be directly declared an instance of \code{Functor}. Because \code{Set} is implemented via balanced
              binary trees, it has an \code{Ord} constraint on the types it can contain. This extra constraint is
              incompatible with the general \code{Functor} definition; we would need \code{fmap} to have the type
              \code{(Ord a, Ord b) => (a -> b) -> f a -> f b}.
        \item \code{Map}, however can be a \code{Functor}. Rather, maps with keys of type \code{k}, i.e., \code{Map k}
              can be a \code{Functor}. Haskell \code{Map}s are represented using balanced binary trees over the
              key type \code{k}, so there is still an \code{Ord} constraint, \code{Functor} cares about the type of
              the values, not the type of the keys.
    \end{notelist}

    \item \textbf{\code{((->) e)}}
    \begin{notelist}
        \item This type looks a big strange. Haskell's syntax does not allow it, but read this type as \code{(e ->)}.
        \item Concretely, the type of the \code{fmap} implementation here would be \code{(a -> b) -> (e -> a) -> (e -> b)}
        \item \cite{typeclassopedia} describes \code{((->) e)} as ``a (possibly infinite) set of values of \code{a}, indexed by
              values of \code{e},'' or ``a context in which a value of type e is available to be consulted in a read-only fashion.''
        \item If we have a predicate \code{isOdd :: Int -> Bool}, and \code{fmap} it over a function \code{length :: String -> Int}
              that returns the length of its parameter, we get a new function of type {String -> Bool} that returns whether or
              not the \code{String}'s length is odd.
        \item From the context perspective, \code{fmap isOdd} takes us from \code{Int}s indexed by \code{String}s to \code{Bool}s
              indexed by \code{String}s.
    \end{notelist}
\end{notelist}

\subsubsection{Functor Laws}

For the Haskell type system, anything that implements \code{fmap :: (a -> b) -> f a -> f b} is perfectly suitable
as an instance of \code{Functor}. However, the concept of functors come to us from the branch of mathematics called
category theory, where functors must satisfy certain laws. In Haskell terms:

\begin{lstlisting}
     fmap id == id
fmap (g . h) == (fmap g) . (fmap h)
\end{lstlisting}

\begin{notelist}
    \item Mapping the identify function over the contents of a \code{Functor} just gives back the original \code{Functor}.
    \item \code{fmap} distributes over function composition.
    \item Ultimately, these two laws just mean that a well-behaved \code{Functor} instance only operates on the
          contents of the \code{Functor}, leaving its structure unchanged.
\end{notelist}

Consider this badly-behaved instance definition for lists taken from \cite{typeclassopedia}.

\begin{lstlisting}
instance [] where
    fmap g []     = []
    fmap g (x:xs) = g x : g x : fmap g xs
\end{lstlisting}

\begin{notelist}
    \item This implementation of \code{fmap} duplicates all the output values: \code{fmap (+1) [1, 2, 3]} returns
          \code{[2, 2, 3, 3, 4, 4]}.
    \item The first law is broken because \code{fmap id [1, 2, 3]} returns \code{[1, 1, 2, 2, 3, 3]} rather than 
          \code{[1, 2, 3]}.
    \item The second law is broken because \code{fmap ((+1) . (*2)) [1,2,3]} returns \code{[3, 3, 5, 5, 7, 7]} rather than
          \code{[3, 5, 7]}.
\end{notelist}

Although Haskell's type system is quite powerful, it is in general undecideable whether a \code{Functor} instance satisfies
the two laws described above, so that requirement cannot be checked at compile time. Since other Haskell code in the standard
libraries and elsewhere will expect new \code{Functor} instances to be well-behaved, it is the responsibility of
the programmer to prove, at least to their own satisfaction, that their implementation satisfies those laws.  We will look at
more typeclasses later on, each with their own laws, and this caveat applies to them as well.
