\section{Monad Transformers}

We have seen how monads describe a general framework for programming with side effects
in an otherwise pure functional language.

However, although we can use \code{IO} for input and output and \code{State} to track mutable state,
we cannot use them together to write code that does both at the same time. We would like to be able
to compose two monads so that we can write code that takes advantage disparate types of side effects.

We will look at one solution to this problem: \textit{monad transformers}. 

Consider the following problem. We want to read lines of text from the console until the user enters
a blank line while keeping a running count the number of lines read and saving the longest line entered so far.

We clearly need \code{IO} to read from the console, but updating the statistics we need seems like
a good use case for \code{State}. Since \code{State} is not strictly necessary, we can implement
this only using only \code{IO}:

\lstinputlisting{code/io-state.hs}

\begin{notelist}
    \item Line 1 introduces a record type to track the statistics we are interested in.
    \item Line 5 defines a function that will begin the process, starting with an initial state.
    \item On line 8 we define a recursive \code{IO} action called \code{runStats}. It takes a \code{Stats}
          value and yields a possibly updated \code{Stats} value.
        \begin{notelist}
            \item We get a line of text from the console, binding the result to the identifier \code{line}.
            \item If the line is empty, we stop and yield the statistics gathered so far. 
            \item In this context, we appear to be using \code{return} like we would in an imperative
                  language. However, in Haskell, \code{return} is just a function that wraps a 
                  value inside a monadic context and has nothing to do with terminating execution
                  of a procedure. We only use \code{return} because we need the result of this
                  expression to be of type \code{IO Stats}.
            \item If the line is not empty, we recursively call \code{runStats} with updated statistics.
        \end{notelist}
    \item Line 15 defines a helper function that takes a \code{String} and  updates a \code{Stats} value 
          accordingly. It increments the count and keeps the given \code{String} if it is longer. Note
          that we are not mutating the original \code{Stats} value, only using its constituents to build
          a new one.
\end{notelist}

This solution works, but it has the drawback that we were responsible for keeping track of the state value,
explicitly passing it through recursive calls to \code{runStats}. We want compose \code{IO} and \code{State}
to give us a monad that gives us mutable state while still allowing the I/O operations we require.

\subsection{The Monad Transformer Library}

The Haskell standard library includes a framework for composing two or more monads, called the Monad Transformer
Library. 

Monad transformers work by layering an interface for monadic operations of one type on top of a base monad.
We saw earlier how to use the \code{State} monad to write stateful code that manipulated a stack of integers.
Here we will extend that example, wrapping the \code{IO} monad with \code{State} to give us state manipulation
and I/O at the same time.

\lstinputlisting[lastline=6]{code/monad-transformers.hs}

\begin{notelist}
    \item We need to import \code{Control.Monad.Trans} to access some basic transformer-related functions, namely \code{lift}
          which we will see below.
    \item The \code{Control.Monad.Trans.State} module gives us the transformer for \code{State}.
    \item On line 4, we define a type synonym for the state values our combined monad will track.
    \item On line 6, we define a type synonym, \code{StatePlusIO}, for our combined monad. 
    \item We build our combined monad using the type \code{StateT s m a}:
        \begin{notelist}
            \item By convention the transformer equivalent of a monad is suffixed with ``\code{T}''. Thus, \code{State}'s
                  transformer equivalent is \code{StateT}.
            \item The first type variable, \code{s}, is the type for the state values, \code{Stack} in the example.
            \item The second type variable, \code{m}, is the type of the underlying monad, \code{IO} in the example.
            \item The third and final type variable, \code{a}, is just the result type of values yielded by operations
                  in the monad. We do not specify it here, since we may want operations of type \code{StatePlusIO Integer}
                  or \code{StatePlusIO ()}, etc.
        \end{notelist}
\end{notelist}

Now we have a new monadic type that has the same stack manipulation categories we saw earlier but with the capability of
performing I/O. We can write some basic actions to interact with this new monad:

\lstinputlisting[firstline=8,lastline=37]{code/monad-transformers.hs}

\begin{notelist}
    \item The functions \code{push} and \code{pop} operate just as they did before, but use \code{lift} to
          turn the \code{IO} action \code{putStrLn} into an action in the \code{StatePlusIO} monad.
    \item The general type of \code{lift} is \code{m a -> t m a}. Haskell can infer both the inner and
          outer monad based on the environment that \code{lift} is used in.
    \item In this case, \code{m} is \code{IO} and \code{t} is \code{StateT Stack}.
    \item So \code{lift} takes \code{putStrLn :: String -> IO ()} and gives us \code{String -> StateT Stack IO ()}.
    \item The \code{readPush} action uses our stateful framework plus I/O to read an integer from the console and
          push the result onto the stack. Again, \code{lift} brings the \code{readLn} action from the underlying
          \code{IO} monad up into our combined monad.
\end{notelist}

Just as with the examples we saw using a simple \code{State} monad, the actions we define for \code{StatePlusIO}
will not do anything until we run them with \code{runStateT}:

\begin{lstlisting}
ghci> runStateT calculator []
? 3
Pushed 3
? 5
Pushed 5
Popped 5
Popped 3
Pushed 8
((),[8])
\end{lstlisting}

To run a \code{State} action, we used \code{runState :: State s a -> s -> (a, s)}, which took a \code{State}
action and an initial state and returned the result along with the final state. The type of \code{runStateT} has a
slightly different type: \code{StateT s m a -> s -> m (a, s)}. We still provide a monadic action (in this case,
built from the \code{StateT} transformer) and an initial state, but the resulting pair is wrapped inside the
underlying monad.

In the case of \code{StatePlusIO}, the result of \code{runStateT} is a value in the \code{IO} monad.

The monad transformer library allows us to compose more than just two monads. In the following example,
we will extend \code{StatePlusIO} with globally accessible, read-only configuration using \code{ReaderT}
to give our stack manipulation code a list of values that it will refuse to store:

\lstinputlisting[lastline=17]{code/more-monad-transformers.hs}

\begin{notelist}
    \item We add an additional import this time, \code{Control.Monad.Trans.Reader} for the \code{Reader}
          monad's transformer equivalent \code{ReaderT}.
    \item We also define a type synonym for our configuration value, in this case a list of integers.
    \item The definition of our new combined type is similar to before, but instead of layering
          \code{StateT} over just \code{IO}, we are building on top of the monad we get from 
          layering \code{ReaderT} over \code{IO}. We are composing three different monads
          to create the combined monad \code{StateReaderIO}.
    \item The \code{push} function receives the biggest change:
        \begin{notelist}
            \item In the \code{Reader} monad, the function \code{ask} yields the read-only value
                  carried along with the computation.
            \item Since \code{ReaderT} is not the top-level monad, we have to use \code{lift} 
                  to bring \code{ask} (an action in the \code{ReaderT Config IO} monad) up into
                  the full \code{StateReaderIO} monad.
            \item We bind this result to the identifier \code{blacklist} and if the value we are
                  attempting to push is in that list, we alert the user that the operation is
                  forbidden. Otherwise, we push the value onto the stack as before.
            \item We use a new function \code{liftIO} to lift the \code{IO} action into our
                  full combined monad. Because \code{lift} can only bring an action up
                  a single level in the monad stack, we would actually need \code{lift . lift}
                  (\code{lift} composed with itself) to bring an \code{IO} action up
                  two levels into \code{StateReaderIO}.
            \item Because \code{IO} is often the base monad on top of which several monad
                  transformers are layered, \code{liftIO} was included in the monad transformer
                  library to lift \code{IO} actions to the top level no matter where in the 
                  stack \code{IO} is actually located.
        \end{notelist}
\end{notelist}

Of course, we need a function to evaluate an action in our combined \code{StateReaderIO} monad.
Recall that \code{runStateT} ran our \code{StateT}-based action and resulted in a value within
the underlying \code{IO} monad.

In this case, \code{runStateT} will take a \code{StateReaderIO} action and an initial state, but
give us a value in the \code{ReaderT Config IO} monad. Thus we need to use \code{runReaderT} to
fully evaluate a \code{StateReaderIO} action to get a result in the \code{IO} monad.

Below we demonstrate \code{readPush} in our new monad, where \code{[1,2,3]} is the blacklist of values
that \code{push} will reject and \code{[]} is the initial state.

\begin{lstlisting}
ghci> runReaderT (runStateT readPush []) [1,2,3]
? 4
Pushed 4
((),[4])

ghci> runReaderT (runStateT readPush []) [1,2,3]
? 2
Forbidden: 2
((),[])
\end{lstlisting}


\subsection{Monadic Parsing}

Parsing text is a frequently encountered problem in computing. Simple recursive descent parsing is almost
identical to computations in the \code{State} monad. A parser for some type \code{t} takes some initial 
state, the input buffer, and returns a value of type \code{t} and the unconsumed input.

However, parsing does not always succeed. For this reason, we would like to extend the \code{State} monad
with the possibility of failure. In this case, we would like failures to come with some explanation. Rather
than use \code{Maybe}, we will use \code{Either e t}, where \code{e} is the type of the error value and
\code{t} is the type of a successful result.

\code{Either} has kind \code{* -> * -> *}, so it cannot be a monad. However, if we apply one of the type
parameters, we get a proper monad: \code{Either e} for some error type \code{e}.

\lstinputlisting{code/either.hs}

\begin{notelist}
    \item \code{Either} has two data constructors, \code{Left} for errors, \code{Right} for values.
    \item The \code{Functor}, \code{Applicative}, and \code{Monad} instances are all analogous to the ones we
          saw for \code{Maybe}, but rather than dealing with \code{Nothing} values, the failure case is  \code{Left}
          wrapping an error value.
    \item We avoid \code{Monad}'s \code{fail} function since it takes a \code{String} and we want to be able
          to use any type for errors. The default implementation (causing a fatal error) will be used. As a result,
          our code will use \code{Right} to signal errors.
\end{notelist}

Just as the \code{State} monad has its transformer equivalent \code{StateT}, \code{Either} has its transformer,
called \code{ExceptT}. We will use \code{ExceptT} on top of \code{State} to build our parsing monad.

\lstinputlisting[lastline=12]{code/parsing.hs}

\begin{notelist}
    \item On line 7, we define \code{ParseError} as a synonym for \code{String}.
    \item On line 9, We define a type synonym for our parser type. \code{Parser a} is much easier to read
          than the alternative, but ultimately our parser type is just stateful computation modified with failure 
          with error messages.
    \item Lines 11-12 define the function to drives our \code{Parser} computations. It takes a parsing computation
          and an initial state and returns the result (a value or error message) plus the unconsumed input. 
\end{notelist}

Now we can write some basic parsing primitives:

\lstinputlisting[firstline=14,lastline=46]{code/parsing.hs}

\begin{notelist}
    \item \code{runParser} is a convenience function that composes running a \code{State} computation with 
          the \code{ExceptT} monad transformer on top of it. 
    \item The function \code{char} tries to parse a single character \code{c}. We \code{get} the current remaining
          parse buffer. If the buffer is empty, our parse fails. If there is at least one character in the buffer, we check 
          whether it is equal to \code{c}. If so, update the state with the remainder of the buffer and yield the character.
          Otherwise, our parse fails.
    \item The \code{string} parser tries to parse a given string. We implement this using the general monadic function
          \code{mapM :: Monad m => (a -> m b) -> [a] -> m [b]}. This function sequences the application of a function
          over each element in a list, yielding the collecting the results. In this case, the function is 
          \code{char :: Char -> Parser Char}. Since \code{String} is equivalent to \code{[Char]}, we sequence
          the parsing of each character in the string, yielding the parsed string. If parsing any of the individual
          characters fails, parsing the entire string fails.
    \item The choice operator \code{(<|>) :: Parser a -> Parser a -> Parser a} attempts the left-hand parser using \code{catchE} which
          runs a computation that might raise an exception and, if an exception occurs, passes the exception value into another action.
          In this case, we ignore the exception value, replace the initial parser state and try the right-hand parser.
    \item \code{many} parses zero or more repetitions of the given parser, yielding a list of the results. It use \code{(<|>)} chooise
          between no parses (\code{return []} on the right) or one or more. This definition uses applicative notation to
          lift \code{(:)} to build lists of parse results in a way that is more succinct than the equivalent \code{do}-notation.
    \item \code{manyOne} is like \code{many}, but requires at least one successful application of the given parser.
\end{notelist}

With these simple primitives and the general library functions we get for \code{Applicative} and \code{Functor} we can
write recursive descent parsers. The following simple parser parses nested parentheses, yielding a tree structure:

\lstinputlisting[firstline=48]{code/parsing.hs}

\begin{lstlisting}
ghci> runParser parens "()"
(Right Leaf,"")

ghci> runParser parens "(()(()))"
(Right (Node [Leaf,Node [Leaf]]),"")

ghci> runParser parens ""
(Left "Unexpected end of input","")
\end{lstlisting}
