\section{Introducing Haskell}
\begin{notelist}
\item Haskell is a \textbf{statically-typed}, \textbf{non-strict}, \textbf{pure} \textbf{functional}
      programming language.
	\begin{notelist}
	\item \textbf{Functional:} Conceptually, computation proceeds via the application of functions to parameters rather than by 
          sequential instructions manipulating values in memory. Functions are first-class values.
	\item \textbf{Pure:} The functions in question are more like mathematical functions than procedures. They map values in an input domain
          to values in an output domain. Pure functions have no side effects. These features are valuable when reasoning about
          and testing our code.
	\item \textbf{Non-strict:} By default, Haskell uses a lazy evaluation strategy. Expressions do not need to be evaluated until the 
          results are needed. For example, Haskell can cleanly represent infinite lists because the language only evaluates such
          an expression as needed.
    \item \textbf{Statically-typed:} The type of every expression is known at compile time, preventing run-time errors caused by type
          incompatibilities. This feature prevents things like Java's \code{NullPointerException}, because a function that claims it returns
          a value of a specific type must live up to that promise. Returning the Haskell equivalent of \code{null} is a compile-time
          type error. Additionally, Haskell makes use of a technique called \textbf{type inference} to figure out the types of most things
          without needing explicit type annotations.
	\end{notelist}
\end{notelist}
\pagebreak

\section{A Sample Program}

\begin{lstlisting}
sumSquares :: Int -> [Int] -> Int
sumSquares count numbers =
    sum (take count (map (^2) numbers))

printSquares :: IO ()
printSquares =
   print $ sumSquares 10 [1..]
\end{lstlisting}

\begin{notelist}
    \item The type signature in line 1 describes a function with two parameters: an \code{Int} and a list
          of \code{Int}s. It returns an \code{Int}. In general, the type after the last \code{->}
          is the return value, and the others are the parameters.
    \item Haskell would actually infer a more general type than what we see in the type annotation 
          on line 1. Haskell programmers usually include type annotations as a form of machine-checked
          documentation.
    \item Haskell functions are defined with an equational syntax as seen in lines 2-3: \code{sumSquares}
          applied to the parameters \code{count} and \code{numbers} is equal to the expression on the
          right-hand side of the equation.
    \item Line 3 shows us several examples of \textbf{function application}. The notation for function application
          is lightweight: simple juxtaposition of terms. The parentheses here are for grouping only. 
    \item Working from the inside out, \code{map} applies a function to each element in a list, producing a new 
          new list containing the resulting values.
    \item \code{(\string^2)} is called a \textbf{section}, a shorthand for an anonymous function whose parameters
          fills in the blank for a binary operator, exponentiation in this case. So \code{map (\string^2)}
          transforms a list of numbers into their squares.
    \item Given a number $n$ and a list, \code{take} returns the first $n$ items of the list, or the whole list
          if it has fewer than $n$ elements. \code{sum} takes a list of numbers and returns the sum.
    \item Lines 5-7 define another function, \code{printSquares}. It takes no parameters, and its return type, \code{IO ()},
          is quite interesting.
    \item Look back at the type \code{[Int]}. The list type \code{[]} is a
          \textbf{parameterized} type. That is, list itself is not a concrete type, but a \textbf{type constructor}.
          We need another type, the parameter, to create a concrete type. Similarly, \code{IO} is a type constructor.
    \item Here we instantiate \code{IO} with the type \code{()}, the \textbf{unit type} with only a single value, also
          written as \code{()}. The type \code{IO ()} represents an \textbf{I/O action} with no interesting return
          value. The \code{IO} type constructor is the technique Haskell uses to perform I/O, inherently impure,
          in a world of pure functions, via a far more general abstraction called the \textbf{monad}.
    \item Consider the expression \code{[1..]}. This expression represents the infinite list of positive
          integers. However, because Haskell evaluates expressions lazily, merely describing the value is not enough to
          force the run-time to evaluate the entire thing. In fact, in this case, evaluation of \code{[1..]} is only
          forced by \code{print} asking for the value produced by \code{sum}, which demands the values yielded by
          \code{take count}, and so on, on demand.
    \item Notice the \code{\$} in line 7. Pronounced ``applied to'', it is an operator with extremely low
          precedence that breaks up the very high precedence of function application, allowing us to avoid nesting
          parentheses. We could have defined \code{sumSquares} using \code{\$} as:
          \code{sum \$ take count \$ map (\string^2) numbers}. 
\end{notelist}
