\section{Monads}

Consider the following Java-like pseudocode:

\begin{lstlisting}
String emailDomain = user.getContactInfo()
                         .getEmailAddress()
                         .getDomain();
\end{lstlisting}

We have several domain objects:

\begin{notelist}
    \item \code{User}, which has method \code{getContactInfo()} that returns the user's contact information with the type 
    \item \code{ContactInfo}, which as a method \code{getEmailAddress()} that returns the associated email address with type
    \item \code{EmailAddress}, which has a method \code{getDomain()} which returns the the domain portion of that email address as a \code{String}.
\end{notelist}

Now suppose that a \code{User}'s \code{ContactInfo} is optional so that \code{getContactInfo()} might return \code{null}. Likewise,
a \code{ContactInfo} record's \code{EmailAddress} is optional so that \code{getEmailAddress()} might return \code{null}.

That makes the above code snippet dangerous. The \code{getEmailAddress()} and \code{getDomain()} calls could be performed on null
references, causing a \code{NullPointerException}. If uncaught, the program crashes.

We could try this:

\begin{lstlisting}
String emailDomain;
ContactInfo contactInfo;
EmailAddress emailAddress;

contactInfo = user.getContactInfo();

if ( contactInfo != null )
{
    emailAddress = contactInfo.getEmailAddress();

    if ( emailAddress != null )
    {
        emailDomain = emailAddress.getDomain();
    }
    else
    {
        emailDomain = null;
    }
}
else
{
    emailDomain = null;
}
\end{lstlisting}

We have managed to propagate possible \code{null} values through the chain of method calls, but at the cost of a great deal of 
boilerplate code.

Of course, we have already seen how Haskell's \code{Maybe} type lets us encode the possibility of \code{null}-like values
explicitly. Suppose we have analagous Haskell types and functions. The above code translates to something like:

\lstinputlisting[firstline=1, lastline=6]{code/maybe-monad.hs}

Although we are now explicit about the possibility of a null result, we have not really addressed the issue of boilerplate.
If we had to chain together even more calls, our code would quickly stair-step right off the screen.

We can see a pattern, however. When we apply \code{getContactInfo} to \code{user}, we do a pattern match. If we got an
actual value (the \code{Just} case), we take that value and pass it on to the \code{getEmailAddress} call. In the \code{null}
case, though, we short-circuit the chain of function calls and just return \code{Nothing}. The same strategy is used when
we try to pass the result of \code{getEmailAddress} into \code{getDomain}.
 
We can generalize this pattern:

\lstinputlisting[firstline=8, lastline=10]{code/maybe-monad.hs}

Note: Here we juse backticks around the function name to cause Haskell treat the function like an infix operator, not unlike
using parentheses around an infix operator causes Haskell to treat it like a regular, prefix function.

Having factored out the pattern matching to handle both cases, we can rewrite our stair-stepped chain of function calls:


\lstinputlisting[firstline=12, lastline=16]{code/maybe-monad.hs}

Now our code is as readable as the original Java chain of method calls, but with the \code{null}-safety of the clunky second attempt.

\subsection{The \code{Monad} Type Class}

Not surprisingly, Haskell offers a type class to describe this pattern in polymorphic terms:

\begin{lstlisting}
class Monad m where
    (>>=)  :: m a -> (a -> m b) -> m b
    (>>)   :: m a -> m b -> m b
    return :: a -> m a 
    fail   :: String -> m a

    m >> k = m >>= \_ -> k
\end{lstlisting}

\begin{notelist}
    \item \code{(>>=)} is the monadic chaining operator. It is a polymorphic, infix equivalent of \code{chainMaybe}, which has the 
          same definition as \code{Maybe}'s implementation of \code{(>>=)}.
    \item \code{(>>)} is a special case of \code{(>>=)} where the value passed into the right-hand function is simply ignored.
          It is given a default implementation in terms of \code{(>>=)} on line 5.
    \item \code{return} is the \code{Monad} class's general method for injecting a value into a monadic wrapper. Code that uses
          the \code{Monad} interface cannot use specific constructors like \code{Just} on line 3 of our final \code{userEmailDomain}
          implementation, so \code{Monad} instances implement \code{return} to define the behavior. If \code{return} sounds familiar,
          it is because it is actually identical to \code{pure} from \code{Applicative}. We will discuss this further in a moment.
    \item \code{fail} offers a way for \code{Monad} instances to short-circuit evaluation when a computation has failed. Use of 
          \code{fail} is discouraged in general because some \code{Monad}s, including \code{IO}, implement \code{fail} by raising a fatal error.
          \code{Maybe}'s implementation of \code{fail} returns \code{Nothing}.
\end{notelist}

\subsection{\code{Monad} and \code{Applicative}}

We mentioned before that \code{return} and \code{pure} were basically identical. In fact, conceptually,
\code{Applicative} is a superclass of \code{Monad}. In fact, the hierarchy from \code{Functor} to \code{Applicative}
to \code{Monad} represents progressively more flexible operations on values inside some context.

However, while \code{Functor} and \code{Monad} were part of the Haskell standard library as described in the Haskell 98 standard \cite{haskell98},
applicative functors were not introduced until 2008 in the paper \emph{Applicative Programming with Effects} \cite{applicative} by McBride and
Paterson.

Altering the standard library's definition of \code{Monad} to include an \code{Applicative} class constraint would break existing user-defined
\code{Monad} instances that did not offer an \code{Applicative} implementation. However, a proposal to make this change is likely to be implemented
in the near future. For that reason, newer versions of GHC will issue warnings when \code{Monad} instances are declared without accompanying
\code{Applicative} instances. 

In the mean time, all the \code{Monad} instances we will discuss have accompanying \code{Applicative}.

\subsection{The List Monad}

Now that we have seen the basic definition of the \code{Monad} type class and seen a simple instance in \code{Maybe}, we
can look at a more complex example: lists.

Recall that lists can be viewed as simple containers or as computational contexts supporting non-deterministic values.
The list monad is based on this non-deterministic value perspective. 

\begin{notelist}
    \item The \code{Functor} instance for lists lifted function application into the domain of non-deterministic values.
    \item The \code{Applicative} instance for lists (i.e., not the \code{ZipList} instance) introduces the ability to
          apply non-deterministic function values to non-deterministic values.
    \item Finally, we can think of the \code{Monad} instance for lists as lifting computation in general into the
          domain of non-deterministic values.
\end{notelist}

Here is how the list instance of \code{Monad} is defined:

\begin{lstlisting}
instance Monad [] where  
    return x = [x]  
    xs >>= f = concat (map f xs)  
    fail _   = []  
\end{lstlisting}

\begin{notelist}
    \item \code{return} is, again, equivalent to \code{pure} and simply returns a singleton list containing the given element.
    \item \code{fail} returns the empty list.
    \item \code{(>>=)} first maps \code{f :: a -> [a]} over \code{xs} yielding a value of type \code{[[a]]}, i.e., a list of lists.
          Then \code{concat :: [[a]] -> [a]} concatenates each of those lists into a single list. For example,
          \code{concat [[1, 2], [3, 4]] == [1, 2, 3, 4]}.
\end{notelist}

Suppose we wanted to work with a square root function that non-deterministically returned both the postive and negative
square roots of its parameter:

\begin{lstlisting}
sqrt' :: Double -> [Double]
sqrt' x = [sqrt x, negate $ sqrt x]
\end{lstlisting}

Now, when we evaluate \code{[4.0, 9.0] >>= sqrt'}, we map \code{sqrt'} over the list, yielding \code{[[2.0, -2.0], [3.0, -3.0]]}. 
Then \code{concat} is applied, yielding \code{[2.0, -2.0, 3.0, -3.0]}.

Suppose we try to evaluate \code{[16.0, 81.0] >>= sqrt' >>= sqrt'} (note that \code{(>>=)} is left associative) so we can simplify
this expression:

\begin{lstlisting}
([16.0, 81.0] >>= sqrt') >>= sqrt' 
== [4.0, -4.0, 9.0, -9.0] >>= sqrt'
== [2.0, -2.0, NaN, NaN, 3.0, -3.0, NaN, NaN]
\end{lstlisting}

When \code{sqrt'} is applied to one of the negative intermediate values, the result is \code{NaN}, since we cannot take the
real square root of a negative. We would like to extend our non-deterministic square root function to deal with that case.

\begin{lstlisting}
sqrt'' :: Double -> [Double]
sqrt'' x | x >= 0.0  = [sqrt x, negate $ sqrt x]
         | otherwise = []
\end{lstlisting}
 
We use \textit{guard clauses} to deal with the two cases. If the parameter is non-negative, yield two possible values, otherwise,
yield no values at all. 

Now, if we evaluate \code{[16.0, 81.0] >>= sqrt'' >>= sqrt''} we get \code{[2.0, -2.0, 3.0, -3.0]}. From the perspective of non-deterministic
values, these are the values that could result from applying \code{sqrt''} twice to the non-deterministic value \code{[16.0, 81.0]}.
When applying \code{sqrt''} to negative values, the return value of \code{[]} represents a path of computation that has failed, and no
trace of it shows up in the final result.

\subsection{The \code{Monad} Laws}

Like \code{Functor} and \code{Applicative}, there are laws that govern how \code{Monad} instances should behave:

\begin{notelist}
    \item \textbf{Left identity}: \code{return a >>= f == f a}
    \item \textbf{Right identity}: \code{m >>= return == m }
    \item \textbf{Associativity}: \code{(m >>= f) >>= g == m >>= (\\x -> f x >>= g x)}
\end{notelist}

The left and right identity laws describe are primarily concerned with the neutral behavior of \code{return}. The associativity
law, with the behavior of sequences of monadic actions linked together with \code{(>>=)}.

Interestingly, none of these laws look much like identity or associativity as we might recall from algebra. However, if we introduce
a new (but related) operator, they do:

\begin{lstlisting}
(>=>) :: Monad m => (a -> m b) -> (b -> m c) -> a -> m c
(f >=> g) x = f x >>= (\y -> g y)
\end{lstlisting}

The operator \code{(>=>)} acts like the standard function composition operator \code{(.) :: (a -> b) -> (b -> c) -> a -> c}
and if we rewrite the above laws in terms of \code{(>=>)}, the names look far more appropriate:

\begin{notelist}
    \item \textbf{Left identity}: \code{return >=> f == f}
    \item \textbf{Right identity}: \code{f >=> return == f }
    \item \textbf{Associativity}: \code{(f >=> g) >=> h == f >=> (g >=> h)}
\end{notelist}

\subsection{The \code{State} Monad}

In the imperative programming paradigm, our programs essentially operate by mutating global state. To swap the values of two variables 
\code{a} and \code{b}, we might store the value in \code{a} into a temporary variable \code{t}, store the value in \code{b} into \code{a},
then store the value in \code{t} into \code{b}. Our programs just shuffle bit patterns around in memory.

In the end, Haskell programs are doing the same thing, but we are interested in expressing our programs in terms of higher-level 
constructs. However, the mutating state is a powerful tool and the \code{State} monad allows us to do emulate this style of
programming by providing the framework through which a value of some type can be passed through a sequence of monadic actions,
possibly being replaced along the way. We do not actually write values to locations in memory, but the end result is similar, while
being built on the same monadic abstraction we have looked at so far.

Here is the definition of the \code{State} type and its \code{Monad} instance.

\begin{lstlisting}
newtype State s a = State { runState :: s -> (a,s) }  

instance Monad (State s) where  
    return x = State $ \s -> (x,s)  
    (State h) >>= f = State $ \s -> let (a, newState) = h s  
                                        (State g) = f a  
                                    in  g newState 
\end{lstlisting}

The \code{State} type constructor takes two type parameters: \code{s} is the type of the state value and \code{a} is
the result type. Just as we might call \code{Maybe Integer} an \code{Integer} value in \code{Maybe}'s context of possible
failure, \code{State String Integer} might be referred to as an \code{Integer} value in the context of a stateful
computation, where the state is a \code{String} value.

The \code{newtype} definition of \code{State} uses Haskell's \textbf{record syntax} to created a \textbf{named field}.
A \code{State} value is actually a wrapper around a function of type \code{s -> (a, s)}. The record syntax automatically
creates a function \code{runState} which pulls that function out of the wrapper so that it can be applied to an initial
state value of type \code{s}, returning the result of type \code{a} and the final state, again of type \code{s}.

The \code{State} type is polymorphic in two type variables, but recall that the definition of the \code{Monad} type class
had only a single type variable. As line 3 suggests, it might be more appropriate to think of it as the \code{State s} monad,
with each concrete type (\code{State [Integer]}, \code{State String}, etc.) as being separate, incompatible monads that
happen to have identical implementations.

Intuitively, a value in the \code{State} monad represents a unit of stateful computation. However, it is important to
emphasize that such a value is fundamentally just a function. Specifically, it is a function that takes
an initial state and performs some computation that results in a value and a new state.

\code{State}'s implementation of \code{return} takes a value \code{x} and returns the simplest \code{State} context possible:
a function that takes any initial state and returns \code{x} along with the initial state unaltered.

We will now consider \code{State}'s implementation of \code{(>>=)}.

The left-hand parameter of \code{(>>=)} is a \code{State} value, and we use pattern matching to bind the wrapped
function to the identifier \code{h}.

The right-hand parameter of \code{(>>=)} is a function \code{f :: a -> State s b}.

We expect the result to be a \code{State} value, and indeed we see an anonymous function wrapped in the \code{State}
constructor.

The anonymous function takes a formal parameter \code{s}, the incoming initial state, and uses \code{let} to make bindings
for some intermediate values. The function \code{h} is applied to the incoming intial state and we bind the resulting pair
to \code{(a, newState)}. We can think of this step as forcing the evaluation of the left-hand stateful computation before
directing the result of that computation into the function on the right-hand side.

The next line binds the result of applying the right-hand side operand, \code{f}, to \code{a}, the result of the left-hand
computation. The result of the expression \code{f a} is a \code{State} value wrapping a function, which is bound to \code{g}.

Finally, our anonymous function will apply \code{g} to {newState}. 

Intuitively, the \code{State} implementation of \code{(>>=)} is the plumbing for a pipeline of computations that take
an initial state and result in a value, along with the final state. Furthermore, any of these \code{State} computations
can be further composed in just the same way.

This is a bit abstract, so an example may be helpful. This example is adapted from a very entertaining introduction
to Haskell called \emph{Learn You a Haskell for Great Good} \cite{learnyou}.

The state that we will use in our computations will be a stack, implemented as \code{[Integer]}. We will define
\code{push} and \code{pop} actions and use them in a small sample program.

\begin{lstlisting}
type Stack = [Integer]

pop :: State Stack Integer
pop = State $ \(x:xs) -> (x, xs)

push :: Integer -> State Stack ()
push x = State $ \xs -> ((), x:xs)
\end{lstlisting}

\begin{notelist}
    \item In line 1, we use \code{type} to define a type synonym.
    \item Consider the type of \code{pop}. The type signature is opaque: \code{pop} is just a stateful computation, working
          with a stack, resulting in an Integer.
    \item The result value is a function. It takes an initial state (our \code{Stack}) and returns a pair: the
          result value and the updated state value. This is exactly the right sort of type signature to wrap
          in the \code{State} constructor.
    \item Note how pattern matching in the anonymous function definition binds the head and tail of the \code{Stack} parameter
          to identifiers we use in the body. However, this pattern match is non-exhaustive (empty lists will not match),
          but we will address that later. This will result in a run-time error, unfortunately. We will address this later.
    \item \code{push} takes an \code{Integer}, the value to be pushed onto the stack, and returns a stateful computation.
          Again, the underlying result is a function wrapped in the \code{State} constructor.
    \item Although it looks strange, recall that \code{()} is the unit type. Its one value (also spelled \code{()}) is used when
          we do not really care about the result value. In this case, the result of a \code{push} is irrelevant, we just want
          the stack to get updated.
\end{notelist}

Now we can use our \code{push} and \code{pop} actions:

\begin{lstlisting}
add :: State Stack ()
add = pop >>= (\x -> pop >>= \y -> (push (x + y)))

simpleMath :: State Stack Integer
simpleMath = push 2 >>
             push 2 >>
             add >>
             pop

result = runState simpleMath []
\end{lstlisting}

\begin{notelist}
    \item We define a new action \code{add} in terms of \code{pop} and \code{push}. In conjunction with \code{(>>=)},
          we use the two anonymous functions to bind the results of the two \code{pop} actions before pushing their sum
          back onto the stack.
    \item In essence, we have extended our language of \code{Stack} actions to include a new stack-based addition operator.
    \item Then in \code{simpleMath}, we use that language to write \code{Stack} manipulation code that has the appearance of
          imperative code. 
\end{notelist}

In order to actually run \code{simpleMath}, we call \code{runState} on the action we want to execute and the initial state.
As described, \code{runState} unwraps the underlying function inside the \code{State} wrapper and that function is immediately
applied to the initial state, giving as a result value and a final state.

\subsection{\code{do}-Notation}

If we look back at the definition of \code{add}, we see a mess of odd-looking operators, anonymous functions, and nested
parentheses. That would be frustrating to deal with and Haskell offers an alternative.

All monads support the use of \textbf{\code{do}-notation}, syntactic sugar that makes monadic code much more readable.
An example:

\begin{lstlisting}
add' = do
    x <- pop
    y <- pop
    push (x + y)

simpleMath' = do
    push' 2
    push' 2
    add'
    pop
\end{lstlisting}

These definitions have exactly the same type and semantics as the previous implementations that used \code{(>>=)} explicitly.

The Haskell compiler simplifies \code{do}-notation systematically, using the following basic patterns:
 

\begin{table}[h]
\begin{tabular}{c c}
    \begin{lstlisting}
    before = do
        op1
    \end{lstlisting}
    &
    \begin{lstlisting}
    after = op1
    \end{lstlisting} \\

    \begin{lstlisting}
    before = do
        op1
        op2
        op3
    \end{lstlisting}
    &
    \begin{lstlisting}
    before = op1 >> 
        do
            op2
            op3

    \end{lstlisting} \\

    \begin{lstlisting}
    before = do
        x <- op1
        op2
        op3
    \end{lstlisting}
    &
    \begin{lstlisting}
    before = op1 >> 
        do
            op2
            op3

    \end{lstlisting} \\
\end{tabular}
\end{table}

\begin{notelist}
    \item Write up left-arrow de-sugaring.
    \item Fix tables.
\end{notelist}

\subsection{The \code{IO} Monad}

As a pure functional language, Haskell functions are unable to have side effects, including modifying program state and I/O.

We have seen how the \code{State} monad allows us to write code that models state manipulation and a similar approach is used
for I/O actions with the \code{IO} monad.

Intuitively, we can think of the \code{IO} monad as a special case of the \code{State} monad where the state value being modified
represents the outside world. An action in the \code{IO} monad that reads a string from the keyboard can be thought of as taking
the current state of the outside world and returning a pair containing the string entered, plus the new, modified state of the
outside world, with the keyboard input consumed.

Just as values of type \code{State a} can be thought of as stateful computations resulting in a value of type \code{a}, values
of type \code{IO a} can be thought of as actions that reach outside of the pure Haskell execution model to perform I/O and
yield a value of type \code{a}. The \code{IO} action described above would have type \code{IO String}. An \code{IO} action
that printed a string to the screen and yielded no interesting result would have type \code{IO ()}.

Recall as well that Haskell is a lazy language. \code{IO} actions are decoupled from the actual execution of the I/O they
describe. For example, in the \code{IO String} action described above, no characters are read from the keyboard until
they are demanded by evaluating an expression that needs them.

Here is an example. We will prompt the user for a file name (trying repeatedly until they entire the name of a file
that exists). Then we will read the contents of the file, process it with a pure function that reverses the individual
words, and print the output to the screen.

\lstinputlisting{code/io.hs}

\begin{notelist}
    \item We begin by importing modules. We need \code{Control.Applicative} for \code{(<\$>)}, \code{System.Directory}
          \code{doesFileExist}.
    \item We begin by defining \code{process}. This is a pure function of type \code{String -> String}.
        \begin{notelist}
            \item \code{process} is written as pipeline of pure functions linked together by the function composition
                  operator \code{(.)}. Data moves through the pipeline from right to left (or, as formatted here,
                  bottom to top).
            \item \code{lines} breaks a \code{String} into a list of \code{String}s on newline characters.
            \item We break each of these lines into its constituent words with \code{map words}.
            \item We now have a list of lists of \code{Strings} (\code{[[String]]}). We reverse each of these
                  individual strings with \code{map (map reverse)}. We need two instances of \code{map} since we are
                  working with nested lists.
            \item With the words reversed, we reassemble the lines with \code{unwords} and reassemble the
                  \code{String} as a whole with \code{unlines}. 
        \end{notelist}
    \item Note how we have separated the pure processing code from the I/O code. We are better able to reason about
          and test \code{process} because we know that the type system will enforce this separation.
    \item \code{reverseWords} is the \code{IO} action that describes the user interaction we want to perform.
        \begin{notelist}
            \item We use \code{do}-notation, which is available for any monad.
            \item The \code{IO} actions \code{putStrLn} and \code{putStr}, both of type \code{String -> IO ()}
                  write a string to the console with and without a newline, respectively.
            \item \code{getLine} reads a line of input from the console. We bind the result to the identifier
                  \code{fileName} with the left-arrow notation.
            \item Now we check whether the file exists. We use \code{doesFileExist :: FilePath -> IO Bool}, where
                  \code{FilePath} is a descriptive type synonym for \code{String}.
            \item We bind the result to the identifier \code{fileExists}. Note that we cannot use the value
                  \code{doesFileExist fileName} directly in an \code{if} statement because it requires a value
                  of type \code{Bool}, not \code{IO Bool}. This restriction is Haskell's type system enforcing
                  the separation of pure and impure code.
            \item If the file does not exist, we print a message and recursively call \code{reverseWords}.
            \item If the file does exist, we will process the contents of the file.
                \begin{notelist}
                    \item The expression \code{readFile fileName} has type \code{IO String}, so we cannot
                          directly apply \code{process} which requires a parameter of type \code{String}.
                    \item However, because \code{IO} is also an instance of \code{Applicative}, we can use
                          \code{(<\$>)} to enable \code{process} to operate on \code{IO String}.
                    \item The expression \code{process <\$> readFile fileName} is itself an \code{IO}
                          action of type \code{IO String}, and we use explicit monadic binding with \code{(>>=)}
                          to pipe the its result into \code{putStr} to write the processed result
                          to the console.
                \end{notelist}
            \item Finally, we define an \code{IO} action \code{main} to just call \code{reverseWords}. When 
                  the Haskell compiler produces an executable, it looks for an action of type \code{IO ()}
                  named \code{main} to be the entry point into the program.
        \end{notelist}
\end{notelist}

\subsection{Conclusion}

We have now looked at a hierarchy of type classes, \code{Functor}, \code{Applicative}, and \code{Monad}. 
These classes successively refine the notion of actions within some computational context.

\code{Functor} offers us the higher-order function \code{fmap} which lifts pure functions into a 
context. We can think of \code{fmap sqrt} as being polymorphic in the type of context it operates on:

\begin{notelist}
    \item In the \code{Maybe} functor, it takes the square root of the number if it exists, and propagates
          the lack of value if not.
    \item In the list functor, it takes the square root of any number of values.
    \item \code{State}, which we have only considered as a monad, is also a \code{Functor}. In this case,
          \code{fmap sqrt} can be applied to a stateful computation, resulting in a new stateful computation
          that leaves the state value unchanged but takes the square root of the number yielded by
          another original stateful computation.
    \item Similarly, in \code{IO}, \code{fmap sqrt} can be applied to an I/O action yielding a number to create
          a new I/O action yielding the square root of that number. So \code{fmap sqrt \$ readLn} is an 
          I/O action that attempts to read a string from the console, parse it as a floating-point number,
          and yields the result.
\end{notelist}

\code{Applicative} extends this idea, with \code{(<*>)} allowing us to use function values that are
themselves inside a computational context. 

Finally, \code{Monad} builds on this notion of computational contexts the notion of chaining together
function applications with \code{do}-notation and the underlying \code{(>>=)} operator. Chaining together
actions within each monad can be seen as programming with a constrained type of side effect.

\begin{notelist}
    \item \code{Maybe} offers computation with the side effect of short-circuit failure: \code{(>>=)} binds
          two actions that might fail. If the first succeeds, the second will be able to operate on the result.
          Otherwise, the entire evaluation fails.
    \item The list monad offers computation with the side effect of non-determinism: \code{(>>=)} binds two
          actions that can each result in many possible values. The second action returns a list containing
          all the results it gets from operating on all the results from the first action.
    \item \code{State} offers computation with the side effect of mutable state: \code{(>>=)} abstracts away
          the process of an action yielding a value along with the current state to a second action,
          simulating a sequence of imperative actions that are mutating some shared state.
    \item Finally, \code{IO} deals with the most general sort of side effect: interaction with the world outside
          Haskell's pure evaluation: binding \code{IO} actions with \code{(>>=)} sequences their execution.
\end{notelist}
     
In the next section, we will look at how we can compose different types of effects using monad transformers.
